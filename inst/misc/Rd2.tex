\documentclass[a4paper]{book}
\usepackage[times,inconsolata,hyper]{Rd}
\usepackage{makeidx}
\usepackage[utf8]{inputenc} % @SET ENCODING@
% \usepackage{graphicx} % @USE GRAPHICX@
\makeindex{}
\begin{document}
\chapter*{}
\begin{center}
{\textbf{\huge Package `wrv'}}
\par\bigskip{\large \today}
\end{center}
\begin{description}
\raggedright{}
\inputencoding{utf8}
\item[Title]\AsIs{Wood River Valley Groundwater-Flow Model}
\item[Version]\AsIs{1.1.1}
\item[Description]\AsIs{A processing program for the groundwater-flow model of the Wood
River Valley aquifer system, south-central Idaho. Included in the package is
MODFLOW-USG version 1.3, a U.S. Geological Survey groundwater-flow model.}
\item[Depends]\AsIs{
R (>= 3.2.0)}
\item[Imports]\AsIs{
dplyr,
grDevices,
graphics,
inlmisc,
methods,
sp,
stats,
raster,
rgdal,
rgeos}
\item[Suggests]\AsIs{
animation,
dataRetrieval,
git2r,
knitr,
leaflet,
sfsmisc,
viridis,
xtable}
\item[SystemRequirements]\AsIs{PEST (>= 13.0, optional)}
\item[License]\AsIs{CC0}
\item[Copyright]\AsIs{This software is in the public domain because it contains materials
that originally came from the United States Geological Survey (USGS), an
agency of the United States Department of Interior. For more information,
see the official USGS copyright policy at
https://www2.usgs.gov/visual-id/credit\_usgs.html}
\item[URL]\AsIs{}\url{https://github.com/USGS-R/wrv}\AsIs{}
\item[BugReports]\AsIs{}\url{https://github.com/USGS-R/wrv/issues}\AsIs{}
\item[Encoding]\AsIs{UTF-8}
\item[ByteCompile]\AsIs{true}
\item[LazyData]\AsIs{true}
\item[LazyDataCompression]\AsIs{xz}
\item[VignetteBuilder]\AsIs{knitr}
\item[RoxygenNote]\AsIs{5.0.1}
\end{description}
\Rdcontents{\R{} topics documented:}
\inputencoding{utf8}
\HeaderA{alluvium.extent}{Extent of Alluvium Unit}{alluvium.extent}
\keyword{datasets}{alluvium.extent}
%
\begin{Description}\relax
Estimated extent of alluvium unit in the Wood River Valley aquifer system.
\end{Description}
%
\begin{Usage}
\begin{verbatim}
alluvium.extent
\end{verbatim}
\end{Usage}
%
\begin{Format}
An object of SpatialPolygonsDataFrame class containing 1 Polygons.
Geographic coordinates are in units of meters, in conformance with the
North American Datum of 1983 (NAD 83), and placed in the
Idaho Transverse Mercator projection (\Rhref{https://www.idwr.idaho.gov/GIS/IDTM/}{IDTM}).
\end{Format}
%
\begin{Source}\relax
Extent defined by Bartollino and Adkins (2012, Plate 1).
\end{Source}
%
\begin{References}\relax
Bartolino, J.R., and Adkins, C.B., 2012,
Hydrogeologic framework of the Wood River Valley aquifer system, south-central Idaho:
U.S. Geological Survey Scientific Investigations Report 2012-5053, 46 p.,
available at \url{https://pubs.usgs.gov/sir/2012/5053/}.
\end{References}
%
\begin{Examples}
\begin{ExampleCode}
sp::plot(alluvium.extent, col = "#BFA76F")
str(alluvium.extent)

\end{ExampleCode}
\end{Examples}
\inputencoding{utf8}
\HeaderA{alluvium.thickness}{Thickness of the Quaternary Sediment}{alluvium.thickness}
\keyword{datasets}{alluvium.thickness}
%
\begin{Description}\relax
Estimated thickness of the Quaternary sediment in the Wood River Valley aquifer system.
\end{Description}
%
\begin{Usage}
\begin{verbatim}
alluvium.thickness
\end{verbatim}
\end{Usage}
%
\begin{Format}
An object of RasterLayer class.
Each cell on the surface grid represents a depth measured from land surface in meters.
Geographic coordinates are in units of meters, in conformance with the
North American Datum of 1983 (NAD 83), and placed in the
Idaho Transverse Mercator projection (\Rhref{https://www.idwr.idaho.gov/GIS/IDTM/}{IDTM}).
The spatial grid is composed of 565 rows and 429 columns,
and has cell sizes that are constant at 100 meters by 100 meters.
\end{Format}
%
\begin{Source}\relax
Revised version of Plate 1 in Bartolino and Adkins (2012).
\end{Source}
%
\begin{References}\relax
Bartolino, J.R., and Adkins, C.B., 2012,
Hydrogeologic framework of the Wood River Valley aquifer system, south-central Idaho:
U.S. Geological Survey Scientific Investigations Report 2012-5053, 46 p.,
available at \url{https://pubs.usgs.gov/sir/2012/5053/}.
\end{References}
%
\begin{Examples}
\begin{ExampleCode}
col <- rainbow(255, start = 0.0, end = 0.8)
raster::image(alluvium.thickness, col = col, asp = 1, axes = FALSE,
              xlab = "", ylab = "")
summary(alluvium.thickness)

\end{ExampleCode}
\end{Examples}
\inputencoding{utf8}
\HeaderA{basalt.extent}{Extent of Basalt Unit}{basalt.extent}
\keyword{datasets}{basalt.extent}
%
\begin{Description}\relax
Estimated extent of the basalt unit underlying the alluvial Wood River Valley aquifer system.
\end{Description}
%
\begin{Usage}
\begin{verbatim}
basalt.extent
\end{verbatim}
\end{Usage}
%
\begin{Format}
An object of SpatialPolygonsDataFrame class containing 1 Polygons.
Geographic coordinates are in units of meters, in conformance with the
North American Datum of 1983 (NAD 83), and placed in the
Idaho Transverse Mercator projection (\Rhref{https://www.idwr.idaho.gov/GIS/IDTM/}{IDTM}).
\end{Format}
%
\begin{Source}\relax
Extent defined by Bartollino and Adkins (2012, Plate 1).
\end{Source}
%
\begin{References}\relax
Bartolino, J.R., and Adkins, C.B., 2012,
Hydrogeologic framework of the Wood River Valley aquifer system, south-central Idaho:
U.S. Geological Survey Scientific Investigations Report 2012-5053, 46 p.,
available at \url{https://pubs.usgs.gov/sir/2012/5053/}.
\end{References}
%
\begin{Examples}
\begin{ExampleCode}
sp::plot(basalt.extent, col = "#BEAED4", border = NA)
sp::plot(alluvium.extent, add = TRUE)
str(basalt.extent)

\end{ExampleCode}
\end{Examples}
\inputencoding{utf8}
\HeaderA{bellevue.wwtp.ponds}{Bellevue Waste Water Treatment Plant Ponds}{bellevue.wwtp.ponds}
\keyword{datasets}{bellevue.wwtp.ponds}
%
\begin{Description}\relax
Location of the Bellevue Waste Water Treatment Plant ponds.
\end{Description}
%
\begin{Usage}
\begin{verbatim}
bellevue.wwtp.ponds
\end{verbatim}
\end{Usage}
%
\begin{Format}
An object of SpatialPolygonsDataFrame class containing 1 Polygons.
Geographic coordinates are in units of meters, in conformance with the
North American Datum of 1983 (NAD 83), and placed in the
Idaho Transverse Mercator projection (\Rhref{https://www.idwr.idaho.gov/GIS/IDTM/}{IDTM}).
\end{Format}
%
\begin{Source}\relax
Idaho Department of Water Resources, accessed on December 11, 2014
\end{Source}
%
\begin{Examples}
\begin{ExampleCode}
sp::plot(bellevue.wwtp.ponds)

\end{ExampleCode}
\end{Examples}
\inputencoding{utf8}
\HeaderA{bypass.canal}{Bypass Canal}{bypass.canal}
\keyword{datasets}{bypass.canal}
%
\begin{Description}\relax
Location of the Bypass Canal in the Wood River Valley.
\end{Description}
%
\begin{Usage}
\begin{verbatim}
bypass.canal
\end{verbatim}
\end{Usage}
%
\begin{Format}
An object of SpatialLines class containing 4 Lines.
Geographic coordinates are in units of meters, in conformance with the
North American Datum of 1983 (NAD 83), and placed in the
Idaho Transverse Mercator projection (\Rhref{https://www.idwr.idaho.gov/GIS/IDTM/}{IDTM}).
\end{Format}
%
\begin{Source}\relax
Idaho Department of Water Resources, accessed on January 15, 2015
\end{Source}
%
\begin{Examples}
\begin{ExampleCode}
sp::plot(bypass.canal)

\end{ExampleCode}
\end{Examples}
\inputencoding{utf8}
\HeaderA{canal.seep}{Canal Seepage}{canal.seep}
\keyword{datasets}{canal.seep}
%
\begin{Description}\relax
Canal seepage as a fraction of diversions for irrigation entities in the Wood River Valley.
\end{Description}
%
\begin{Usage}
\begin{verbatim}
canal.seep
\end{verbatim}
\end{Usage}
%
\begin{Format}
An object of class data.frame with 19 records and the following variables:
\begin{description}

\item[EntityName] name of the irrigation entity served by the canal system.
\item[SeepFrac] estimated canal seepage as a fraction of diversions.

\end{description}
\end{Format}
%
\begin{Source}\relax
Idaho Department of Water Resources, accessed on November 4, 2015
\end{Source}
%
\begin{SeeAlso}\relax
\code{\LinkA{canals}{canals}}
\end{SeeAlso}
%
\begin{Examples}
\begin{ExampleCode}
str(canal.seep)

d <- canal.seep[order(canal.seep$SeepFrac, decreasing=TRUE), ]
par(mar = c(4.1, 8.1, 0.1, 0.6))
barplot(d$SeepFrac, names.arg = d$EntityName, horiz = TRUE, cex.names = 0.7,
        cex.axis = 0.7, cex.lab = 0.7, las = 1, xlab = "Seepage fraction")

graphics.off()

\end{ExampleCode}
\end{Examples}
\inputencoding{utf8}
\HeaderA{canals}{Canal Systems}{canals}
\keyword{datasets}{canals}
%
\begin{Description}\relax
Canal systems in the Wood River Valley and surrounding areas.
\end{Description}
%
\begin{Usage}
\begin{verbatim}
canals
\end{verbatim}
\end{Usage}
%
\begin{Format}
An object of SpatialLinesDataFrame class containing
113 Lines and a data.frame with the following variable:
\begin{description}

\item[EntityName] name of the irrigation entity served by the canal system.
\item[Name] local canal name

\end{description}
\end{Format}
%
\begin{Source}\relax
Idaho Department of Water Resources, accessed on November 29, 2014
\end{Source}
%
\begin{SeeAlso}\relax
\code{\LinkA{r.canals}{r.canals}}, \code{\LinkA{canal.seep}{canal.seep}}
\end{SeeAlso}
%
\begin{Examples}
\begin{ExampleCode}
sp::plot(canals, col = "#3399CC")
str(canals@data)

\end{ExampleCode}
\end{Examples}
\inputencoding{utf8}
\HeaderA{cities}{Cities and Towns}{cities}
\keyword{datasets}{cities}
%
\begin{Description}\relax
Cities and towns in the Wood River Valley and surrounding areas.
\end{Description}
%
\begin{Usage}
\begin{verbatim}
cities
\end{verbatim}
\end{Usage}
%
\begin{Format}
An object of SpatialPointsDataFrame class containing 11 points.
Geographic coordinates are in units of meters, in conformance with the
North American Datum of 1983 (NAD 83), and placed in the
Idaho Transverse Mercator projection (\Rhref{https://www.idwr.idaho.gov/GIS/IDTM/}{IDTM}).
\end{Format}
%
\begin{Source}\relax
Idaho Department of Water Resources
(\Rhref{https://research.idwr.idaho.gov/index.html#GIS-Data}{IDWR}),
accessed on April 15, 2015
\end{Source}
%
\begin{Examples}
\begin{ExampleCode}
str(cities)

col <- "#333333"
sp::plot(cities, pch = 15, cex = 0.8, col = col)
text(cities, labels = cities@data$FEATURE_NA, col = col, cex = 0.5, pos = 1, offset = 0.4)

\end{ExampleCode}
\end{Examples}
\inputencoding{utf8}
\HeaderA{clay.extent}{Extent of Clay Unit}{clay.extent}
\keyword{datasets}{clay.extent}
%
\begin{Description}\relax
Estimated extent of the clay confining unit (aquitard) separating the
unconfined aquifer from the underlying confined aquifer in the
Wood River Valley aquifer system.
\end{Description}
%
\begin{Usage}
\begin{verbatim}
clay.extent
\end{verbatim}
\end{Usage}
%
\begin{Format}
An object of SpatialPolygonsDataFrame class containing 2 Polygons.
Geographic coordinates are in units of meters, in conformance with the
North American Datum of 1983 (NAD 83), and placed in the
Idaho Transverse Mercator projection (\Rhref{https://www.idwr.idaho.gov/GIS/IDTM/}{IDTM}).
\end{Format}
%
\begin{Source}\relax
Extent defined by Moreland (1977, fig. 3 in USGS Open-File report).
Moreland (1977) shows an outlier by Picabo that is assumed to
indicate confined conditions in the basalt and not the lake sediments.
\end{Source}
%
\begin{References}\relax
Moreland, J.A., 1977, Ground water-surface water relations in the Silver Creek area,
Blaine County, Idaho: Boise, Idaho Department of Water Resources, Water Information Bulletin 44,
42 p., 5 plates in pocket, accessed January 31, 2012.
Also published as U.S. Geological Survey Open-File report 77-456, 66 p.,
available at \url{https://pubs.er.usgs.gov/publication/ofr77456}.
\end{References}
%
\begin{Examples}
\begin{ExampleCode}
sp::plot(clay.extent, col = "#FDC086", border = NA)
sp::plot(alluvium.extent, add = TRUE)
str(clay.extent)

\end{ExampleCode}
\end{Examples}
\inputencoding{utf8}
\HeaderA{comb.sw.irr}{Combined Surface-Water Irrigation Diversions}{comb.sw.irr}
\keyword{datasets}{comb.sw.irr}
%
\begin{Description}\relax
Supplemental groundwater rights and associated surface-water rights.
\end{Description}
%
\begin{Usage}
\begin{verbatim}
comb.sw.irr
\end{verbatim}
\end{Usage}
%
\begin{Format}
An object of class data.frame with 1,213 records and the following variables:
\begin{description}

\item[WaterRight] name of the supplemental groundwater right.
\item[CombWaterRight] name of the surface-water right that shares a
combined limit with the groundwater right.
\item[Source] river or stream source name for the surface-water right.
\item[WaterUse] authorized beneficial use for the surface-water right.
\item[MaxDivRate] authorized maximum diversion rate for the surface-water right,
in cubic meters per day.
\item[Pdate] priority date of the surface-water right.

\end{description}
\end{Format}
%
\begin{Source}\relax
Idaho Department of Water Resources (IDWR), accessed on April 25, 2014;
derived from combined limit comments in IDWR water rights database.
\end{Source}
%
\begin{Examples}
\begin{ExampleCode}
str(comb.sw.irr)

\end{ExampleCode}
\end{Examples}
\inputencoding{utf8}
\HeaderA{div.gw}{Groundwater Diversions}{div.gw}
\keyword{datasets}{div.gw}
%
\begin{Description}\relax
Groundwater diversions recorded by Water District 37 or municipal water providers.
Groundwater is diverted from the aquifer by means of either pumping wells or
flowing-artesian wells.
\end{Description}
%
\begin{Usage}
\begin{verbatim}
div.gw
\end{verbatim}
\end{Usage}
%
\begin{Format}
An object of class data.frame with 7,292 records and the following variables:
\begin{description}

\item[YearMonth] year and month during which diversions were recorded,
with a required date format of YYYYMM.
\item[Diversion] name of the well
\item[Reach] name of the river subreach into which the well water is discharged;
only applicable to exchange wells.
\item[BigReach] name of the river reach into which the well water is discharged;
only applicable to exchange wells.
\item[EntityName] name of the irrigation entity which the well supplies water.
\item[WMISNumber] well number in the Idaho Department of Water Resources (IDWR)
Water Measurement Information System.
\item[GWDiv] volume of water diverted during the month, in cubic meters.

\end{description}
\end{Format}
%
\begin{Source}\relax
Idaho Department of Water Resources (IDWR), accessed on December 11, 2014;
compiled data records from Water District 37 and 37M, City of Ketchum,
Sun Valley Water and Sewer District, City of Hailey, and City of Bellevue.
\end{Source}
%
\begin{Examples}
\begin{ExampleCode}
str(div.gw)

\end{ExampleCode}
\end{Examples}
\inputencoding{utf8}
\HeaderA{div.ret.exch}{Diversions, Returns, and Exchange Wells}{div.ret.exch}
\keyword{datasets}{div.ret.exch}
%
\begin{Description}\relax
Location of streamflow diversions, irrigation canal or pond returns, and
exchange well returns.
\end{Description}
%
\begin{Usage}
\begin{verbatim}
div.ret.exch
\end{verbatim}
\end{Usage}
%
\begin{Format}
An object of SpatialPointsDataFrame class containing 117 points with the
following variables:
\begin{description}

\item[Name] local name for the diversion/return site.
\item[Type] data type, either ``Diversion'', ``Return'', or
``Exchange well inflow''.
\item[LocSource] data source
\item[Big] corresponding river reach

\end{description}

Geographic coordinates are in units of meters, in conformance with the
North American Datum of 1983 (NAD 83), and placed in the
Idaho Transverse Mercator projection (\Rhref{https://www.idwr.idaho.gov/GIS/IDTM/}{IDTM}).
\end{Format}
%
\begin{Source}\relax
Idaho Department of Water Resources, accessed on June 5, 2015
\end{Source}
%
\begin{Examples}
\begin{ExampleCode}
sp::plot(div.ret.exch)
str(div.ret.exch@data)

\end{ExampleCode}
\end{Examples}
\inputencoding{utf8}
\HeaderA{div.sw}{Surface-Water Diversions}{div.sw}
\keyword{datasets}{div.sw}
%
\begin{Description}\relax
Surface-water diversions recorded by Water District 37 or municipal water providers.
\end{Description}
%
\begin{Usage}
\begin{verbatim}
div.sw
\end{verbatim}
\end{Usage}
%
\begin{Format}
An object of class data.frame with 15,550 records and the following variables:
\begin{description}

\item[YearMonth] year and month during which diversions were recorded,
with a required date format of YYYYMM.
\item[Diversion] name of the surface-water diversion.
\item[Reach] river subreach from which the water is diverted.
\item[BigReach] river reach from which the water is diverted.
\item[EntityName] name of the irrigation entity which the diversion supplies water.
\item[SWDiv] volume of water diverted during the month, in cubic meters.

\end{description}
\end{Format}
%
\begin{Source}\relax
Idaho Department of Water Resources, accessed on December 11, 2014;
compiled data records from Water District 37 and 37M, City of Hailey,
City of Bellevue, City of Ketchum, and Sun Valley Water and Sewer District.
\end{Source}
%
\begin{Examples}
\begin{ExampleCode}
str(div.sw)

\end{ExampleCode}
\end{Examples}
\inputencoding{utf8}
\HeaderA{div.ww}{Wastewater Treatment Plant Diversions}{div.ww}
\keyword{datasets}{div.ww}
%
\begin{Description}\relax
Discharge from wastewater treatment plants.
\end{Description}
%
\begin{Usage}
\begin{verbatim}
div.ww
\end{verbatim}
\end{Usage}
%
\begin{Format}
An object of class data.frame with 1,182 records and the following variables:
\begin{description}

\item[YearMonth] year and month during which diversions were recorded,
with a required date format of YYYYMM.
\item[Return] name of the wastewater treatment plant.
\item[Reach] name of the river subreach to which treated effluent is discharged;
only applicable to wastewater treatment plants that discharge to the river.
\item[BigReach] name of the river reach to which treated effluent is discharged;
only applicable to wastewater treatment plants that discharge to the river.
\item[EntityName] name of the irrigation entity served by the wastewater treatment plant.
\item[WWDiv] volume of wastewater discharged during the month, in cubic meters.

\end{description}
\end{Format}
%
\begin{Source}\relax
Idaho Department of Water Resources and U.S. Geological Survey,
accessed on August 11, 2014; compiled data records from the
U.S. Environmental Protection Agency for plants that discharge to the river,
and from records of the Idaho Department of Environmental Quality for
plants that discharge to land application.
\end{Source}
%
\begin{Examples}
\begin{ExampleCode}
str(div.ww)

\end{ExampleCode}
\end{Examples}
\inputencoding{utf8}
\HeaderA{drains}{Drain Boundaries at Stanton Crossing and Silver Creek}{drains}
\keyword{datasets}{drains}
%
\begin{Description}\relax
Polygons used to define the locations of drain boundaries in the model domain.
The polygons clip the line segments along the aquifer boundary (see \code{\LinkA{alluvium.extent}{alluvium.extent}}),
and model cells intersecting these clipped-line segments are defined as boundary cells.
\end{Description}
%
\begin{Usage}
\begin{verbatim}
drains
\end{verbatim}
\end{Usage}
%
\begin{Format}
An object of SpatialPolygonsDataFrame class containing a set of 2 Polygons and
a data.frame with the following variable:
\begin{description}

\item[Name] identifier for the polygon.
\item[cond] drain conductance in square meters per day.
\item[elev] drain threshold elevation in meters above the
North American Vertical Datum of 1988 (NAVD 88).

\end{description}

Geographic coordinates are in units of meters, in conformance with the
North American Datum of 1983 (NAD 83), and placed in the
Idaho Transverse Mercator projection (\Rhref{https://www.idwr.idaho.gov/GIS/IDTM/}{IDTM}).
\end{Format}
%
\begin{Source}\relax
U.S. Geological Survey, accessed on March 27, 2015;
a Keyhole Markup Language (\Rhref{https://en.wikipedia.org/wiki/Kml}{KML}) file created in
\Rhref{https://www.google.com/earth/}{Google Earth} with polygons drawn by hand in
areas of known drains.
\end{Source}
%
\begin{Examples}
\begin{ExampleCode}
str(drains)

sp::plot(drains, border = "red")
sp::plot(alluvium.extent, add = TRUE)

\end{ExampleCode}
\end{Examples}
\inputencoding{utf8}
\HeaderA{drybed}{Dry River Bed and Stream Fed Creek Conditions}{drybed}
\keyword{datasets}{drybed}
%
\begin{Description}\relax
A summary of dry river bed and stream fed conditions in the Wood River Valley, Idaho.
Stream reaches on the Big Wood River between Glendale and Wood River Ranch are episodically dry;
these dry periods are specified for calendar months when water diversions to the
Bypass Canal begins before the 16th of the month and ends after the 15th of the month.
\end{Description}
%
\begin{Usage}
\begin{verbatim}
drybed
\end{verbatim}
\end{Usage}
%
\begin{Format}
An object of data.frame class with 12 records and the following variables:
\begin{description}

\item[Reach] stream reach name
\item[199501,\dots,201012] logical values indicating whether the stream reach exhibits
dry-bed conditions during a stress period.

\end{description}
\end{Format}
%
\begin{Source}\relax
Idaho Department of Water Resources, accessed on January 6, 2016;
compiled from Water District 37 records.
\end{Source}
%
\begin{Examples}
\begin{ExampleCode}
str(drybed)

\end{ExampleCode}
\end{Examples}
\inputencoding{utf8}
\HeaderA{efficiency}{Irrigation Efficiency}{efficiency}
\keyword{datasets}{efficiency}
%
\begin{Description}\relax
The irrigation efficiency of irrigation entities in the Wood River Valley, Idaho.
\end{Description}
%
\begin{Usage}
\begin{verbatim}
efficiency
\end{verbatim}
\end{Usage}
%
\begin{Format}
An object of data.frame class with 88 records and the following variables:
\begin{description}

\item[EntityName] name of the irrigation entity which the irrigation efficiency is applied.
\item[Eff] estimated irrigation efficiency, the ratio of the amount of
water consumed by the crop to the amount of water supplied through irrigation.
\item[Comment] brief comment on irrigation conditions.

\end{description}
\end{Format}
%
\begin{Source}\relax
Idaho Department of Water Resources, accessed on July 9, 2015
\end{Source}
%
\begin{Examples}
\begin{ExampleCode}
str(efficiency)

\end{ExampleCode}
\end{Examples}
\inputencoding{utf8}
\HeaderA{entity.components}{Irrigation Entity Components}{entity.components}
\keyword{datasets}{entity.components}
%
\begin{Description}\relax
Irrigation entities and their components in the Wood River Valley and surrounding areas.
An irrigation entity is defined as an area served by a group of surface-water and/or
groundwater diversion(s).
\end{Description}
%
\begin{Usage}
\begin{verbatim}
entity.components
\end{verbatim}
\end{Usage}
%
\begin{Format}
An object of list class with components of SpatialPolygonsDataFrame-class.
There are a total of 192 components, one for each month in the 1995--2010 time period.
Linked data.frame objects have the following variables:
\begin{description}

\item[EntitySrce] concatenation of the \code{EntityName} and \code{Source} character strings.
\item[mean.et] mean evapotranspiration (ET) on irrigated and semi-irrigated lands in meters.
\item[area] area of irrigated and semi-irrigated lands in square meters.
\item[PrecipZone] name of the precipitation zone;
see \code{\LinkA{precip.zones}{precip.zones}} dataset for details.
\item[et.vol] volume of ET on irrigated and semi-irrigated lands in cubic meters.
\item[precip.vol] volume of precipitation on irrigated and semi-irrigated lands in cubic meters.
\item[cir.vol] volume of crop irrigation requirement in cubic meters (ET minus precipitation).
\item[EntityName] name of the irrigation entity.
\item[Source] water source, either ``Mixed'' for a mixture of surface water and groundwater,
``SW Only'' for surface water only, or ``GW Only'' for groundwater only.

\end{description}
\end{Format}
%
\begin{Source}\relax
Calculated from the \code{\LinkA{irr.entities}{irr.entities}}, \code{\LinkA{wetlands}{wetlands}},
\code{\LinkA{public.parcels}{public.parcels}}, \code{\LinkA{irr.lands.year}{irr.lands.year}}, \code{\LinkA{et}{et}}, and
\code{\LinkA{precipitation}{precipitation}} datasets;
see the \file{package-datasets} vignette for the \R{} code used in this calculation.
\end{Source}
%
\begin{Examples}
\begin{ExampleCode}
names(entity.components)
sp::plot(entity.components[["199506"]])
print(entity.components[["199506"]])

\end{ExampleCode}
\end{Examples}
\inputencoding{utf8}
\HeaderA{et}{Evapotranspiration}{et}
\keyword{datasets}{et}
%
\begin{Description}\relax
Evapotranspiration (ET) in the Wood River Valley and surrounding areas.
Defined as the amount of water lost to the atmosphere via direct evaporation,
transpiration by vegetation, or sublimation from snow covered areas.
\end{Description}
%
\begin{Usage}
\begin{verbatim}
et
\end{verbatim}
\end{Usage}
%
\begin{Format}
An object of RasterStack class containing 192 RasterLayer objects,
one layer for each month in the 1995-2010 time period.
Each cell on a layers surface grid represents the monthly depth of ET in meters.
Geographic coordinates are in units of meters, in conformance with the
North American Datum of 1983 (NAD 83), and placed in the
Idaho Transverse Mercator projection (\Rhref{https://www.idwr.idaho.gov/GIS/IDTM/}{IDTM}).
\end{Format}
%
\begin{Source}\relax
Idaho Department of Water Resources, accessed on November 17, 2014
\end{Source}
%
\begin{SeeAlso}\relax
\code{\LinkA{et.method}{et.method}}
\end{SeeAlso}
%
\begin{Examples}
\begin{ExampleCode}
sp::plot(et[["199505"]])
print(et)

\end{ExampleCode}
\end{Examples}
\inputencoding{utf8}
\HeaderA{et.method}{Method Used to Calculate Evapotranspiration}{et.method}
\keyword{datasets}{et.method}
%
\begin{Description}\relax
Methods used to estimate monthly distributions of evapotranspiration (ET) rate.
\end{Description}
%
\begin{Usage}
\begin{verbatim}
et.method
\end{verbatim}
\end{Usage}
%
\begin{Format}
An object of data.frame class with 122 records with the following variables:
\begin{description}

\item[YearMonth] year and month during which the method was applied,
with a required date format of YYYYMM.
\item[ETMethod] Identifier that indicates the method used to estimate ET values.
Identifiers include either
``Allen-Robison'', the Allen and Robison method (Allen and Robison, 2007);
``METRIC'', the Mapping Evapotranspiration at high Resolution and with
Internalized Calibration (METRIC) model (Allen and others, 2010a);
``NDVI'', the Normalized Difference Vegetation Index (NDVI) method
(Allen and others, 2010b);
``Interpolation'', interpolation from known ET data; or
``METRIC-NDVI'', a combination of METRIC and NDVI methods.

\end{description}
\end{Format}
%
\begin{Source}\relax
Idaho Department of Water Resources, accessed on April 27, 2015
\end{Source}
%
\begin{References}\relax
Allen, R., and Robison, C.W., 2007, Evapotranspiration and
consumptive water requirements for Idaho, University of Idaho, Kimberly, Idaho.

Allen, R., Tasumi, M., Trezza, R., and Kjaersgaard, J., 2010a,
METRIC mapping evapotranspiration at high resolution applications manual for
Landsat satellite imagery version 2.07, University of Idaho, Kimberly, ID.

Allen, R., Robison, C.W., Garcia, M., Trezza, R., Tasumi, M., and Kjaersgaard, J., 2010b,
ETrF vs NDVI relationships for southern Idaho for rapid estimation of evapotranspiration,
University of Idaho, Kimberly, ID.

ET Idaho: \url{http://data.kimberly.uidaho.edu/ETIdaho/}
\end{References}
%
\begin{Examples}
\begin{ExampleCode}
str(et.method)

\end{ExampleCode}
\end{Examples}
\inputencoding{utf8}
\HeaderA{gage.disch}{Daily Mean Discharge at Streamgages}{gage.disch}
\keyword{datasets}{gage.disch}
%
\begin{Description}\relax
The daily mean discharge at streamgages in the Big Wood River Valley, Idaho.
Discharge records bracket the 1992-2014 time period and are based on
records with quality assurance code of approved (`A').
\end{Description}
%
\begin{Usage}
\begin{verbatim}
gage.disch
\end{verbatim}
\end{Usage}
%
\begin{Format}
An object of data.frame class with 8,315 records and the following variables:
\begin{description}

\item[Date] date during which discharge was averaged.
\item[13135500] daily mean discharge in cubic meters per day, recorded at the USGS
\Rhref{https://waterdata.usgs.gov/id/nwis/uv/?site_no=13135500}{13135500}
Big Wood River near Ketchum streamgage.
\item[13139510] daily mean discharge in cubic meters per day, recorded at the USGS
\Rhref{https://waterdata.usgs.gov/id/nwis/uv/?site_no=13139510}{13139510}
Big Wood River at Hailey streamgage.
\item[13140800] daily mean discharge in cubic meters per day, recorded at the USGS
\Rhref{https://waterdata.usgs.gov/id/nwis/uv/?site_no=13140800}{13140800}
Big Wood River at Stanton Crossing near Bellevue streamgage.

\end{description}
\end{Format}
%
\begin{Source}\relax
National Water Information System (\Rhref{https://waterdata.usgs.gov/nwis}{NWIS}),
accessed on January 8, 2015
\end{Source}
%
\begin{Examples}
\begin{ExampleCode}
str(gage.disch)

col <- c("red", "blue", "green")
ylab <- paste("Discharge in cubic", c("meters per day", "acre-foot per year"))
inlmisc::PlotGraph(gage.disch, ylab = ylab, col = col, lty = 1:3,
                   conversion.factor = 0.29611)
leg <- sprintf("USGS \%s", names(gage.disch)[-1])
legend("topright", leg, col = col, lty = 1:3, inset = 0.02, cex = 0.7,
       box.lty = 1, bg = "#FFFFFFE7")

graphics.off()

\end{ExampleCode}
\end{Examples}
\inputencoding{utf8}
\HeaderA{gage.height}{Daily Mean Gage Height at Streamgages}{gage.height}
\keyword{datasets}{gage.height}
%
\begin{Description}\relax
The daily mean gage height at streamgages in the Big Wood River Valley, Idaho.
Gage height records bracket the 1987-2014 and are based on records with
quality assurance codes of working (`W'), in review (`R'), and
approved (`A').
\end{Description}
%
\begin{Usage}
\begin{verbatim}
gage.height
\end{verbatim}
\end{Usage}
%
\begin{Format}
An object of data.frame class with 9,980 records and the following variables:
\begin{description}

\item[Date] date during which gage height was averaged.
\item[13135500] daily mean gage height in meters, recorded at the USGS
\Rhref{https://waterdata.usgs.gov/id/nwis/uv/?site_no=13135500}{13135500}
Big Wood River near Ketchum streamgage.
\item[13139510] daily mean gage height in meters, recorded at the USGS
\Rhref{https://waterdata.usgs.gov/id/nwis/uv/?site_no=13139510}{13139510}
Big Wood River at Hailey streamgage.
\item[13140800] daily mean gage height in meters, recorded at the USGS
\Rhref{https://waterdata.usgs.gov/id/nwis/uv/?site_no=13140800}{13140800}
Big Wood River at Stanton Crossing near Bellevue streamgages.

\end{description}
\end{Format}
%
\begin{Source}\relax
Data queried from the National Water Information System
(\Rhref{https://waterdata.usgs.gov/nwis}{NWIS}) database on December 15, 2014,
by Ross Dickinson (USGS).
Records recorded on May 26-28, 1991 and March 15-22, 1995 were reassigned
quality assurance codes of `I' because of assumed ice build-up.
Missing data at the Big Wood River near Ketchum gage was estimated using a
linear regression model based on recorded gage-height data at the Big Wood River at
Hailey and Near Ketchum streamgages.
Missing data at the Stanton Crossing near Bellevue gage was replaced with
average gage-height values recorded at this gage.
\end{Source}
%
\begin{Examples}
\begin{ExampleCode}
str(gage.height)

col <- c("red", "blue", "green")
ylab <- paste("Gage height in", c("meters", "feet"))
inlmisc::PlotGraph(gage.height, ylab = ylab, col = col, lty = 1:3,
                   conversion.factor = 3.28084)
leg <- sprintf("USGS \%s", names(gage.height)[-1])
legend("topright", leg, col = col, lty = 1:3, inset = 0.02, cex = 0.7,
       box.lty = 1, bg = "#FFFFFFE7")

graphics.off()

\end{ExampleCode}
\end{Examples}
\inputencoding{utf8}
\HeaderA{GetSeasonalMult}{Get Seasonal Multiplier}{GetSeasonalMult}
\keyword{manip}{GetSeasonalMult}
%
\begin{Description}\relax
This function determines the seasonal fraction of the long-term mean value.
\end{Description}
%
\begin{Usage}
\begin{verbatim}
GetSeasonalMult(x, reduction, d.in.mv.ave, fixed.dates)
\end{verbatim}
\end{Usage}
%
\begin{Arguments}
\begin{ldescription}
\item[\code{x}] data.frame.
Time series data (observations) with components of class Date and numeric.

\item[\code{reduction}] numeric.
Signal amplitude reduction, a dimensionless quantity.
Its magnitude should be greater than or equal to 1;
where a value of 1 indicates no reduction in the signal amplitude.

\item[\code{d.in.mv.ave}] numeric.
Number of days in the moving average subset.

\item[\code{fixed.dates}] Date.
Vector of equally spaced dates, these are the fixed locations where the moving average is calculated.
The final date is neglected.
\end{ldescription}
\end{Arguments}
%
\begin{Details}\relax
A simple moving average is first calculated at dates specified in \code{fixed.dates}
using past observational data in \code{x}
(such as the previous 9-months of stage data recorded at a streamgage).
The seasonal average of the moving average is then passed through a signal amplitude reduction algorithm.
The reduced values are then divided by the mean of the seasonal reduced data to give
the seasonal fraction of the mean (also known as the seasonal scaling index).
\end{Details}
%
\begin{Value}
Returns an object of class data.frame with the following variables:
\begin{description}

\item[\code{names(x)[1]}] start date for each season in \code{fixed.dates}.
\item[multiplier] seasonal scaling index

\end{description}

\end{Value}
%
\begin{Author}\relax
J.C. Fisher, U.S. Geological Survey, Idaho Water Science Center

A.H. Wylie and J. Sukow, Idaho Department of Water Resources
\end{Author}
%
\begin{Examples}
\begin{ExampleCode}
obs <- dataRetrieval::readNWISdata(sites = "13139510", parameterCd = "00060",
                                   startDate = "1992-01-01", endDate = "2011-01-01")
obs <- obs[, c("dateTime", "X_00060_00003")]
obs[, 1] <- as.Date(obs[, 1])

fixed.dates <- seq(as.Date("1995-01-01"), as.Date("2011-01-01"), "1 month")
d <- GetSeasonalMult(obs, 2, 273.932, fixed.dates)
str(d)

\end{ExampleCode}
\end{Examples}
\inputencoding{utf8}
\HeaderA{GetWellConfig}{Get Well Completion and Pumping Rate in Model Space}{GetWellConfig}
\keyword{manip}{GetWellConfig}
%
\begin{Description}\relax
This function determines well completions and pumping rates in model space.
The pumping rate is specified for each model cell intersecting a well's open interval(s)
and calculated by multiplying the estimated pumping demand by the cell's transmissivity fraction.
The transmissivity fraction is calculated by dividing the cell's aquifer transmissivity by
the sum of all transmissivity values for cells belonging to the same well.
The transmissivity fraction calculation assumes saturated conditions in the model cell.
\end{Description}
%
\begin{Usage}
\begin{verbatim}
GetWellConfig(rs.model, wells, well.col, rate.col = NULL,
  lay2.hk.tol = 0.01)
\end{verbatim}
\end{Usage}
%
\begin{Arguments}
\begin{ldescription}
\item[\code{rs.model}] RasterStack.
Composed of raster layers describing the model grid and hydraulic conductivity distribution:
\code{lay1.top}, \code{lay1.bot}, \code{lay2.bot}, \code{lay3.bot},
\code{lay1.hk}, \code{lay2.hk}, and \code{lay3.hk}.

\item[\code{wells}] SpatialPointsDataFrame.
Average pumping rate for each well during various times.

\item[\code{well.col}] character.
Column name of the well identifier field.

\item[\code{rate.col}] character.
Vector of column names for the pumping rate fields.

\item[\code{lay2.hk.tol}] numeric.
Hydraulic conductivity tolerance for model cells in layer 2.
Used to prevent pumping in the aquitard layer of the aquifer system.
Pumping is prohibited in model layer 2 cells with hydraulic conductivity values less than
\code{lay2.hk.tol} and a well opening isolated to layer 2;
for these cases, pumping is allocated to the adjacent layer 1 cell.
\end{ldescription}
\end{Arguments}
%
\begin{Value}
Returns an object of class data.frame with the following components:
\begin{description}

\item[\dots] unique identifier assigned to a well, its name is specified by \code{well.col}.
\item[lay,row,col] layer, row, and column number of a model cell, respectively.
\item[hk] hydraulic conductivity of the model cell, in meters per day.
\item[thk] vertical length of the well opening (open borehole or screen) in the model cell, in meters.
A value of zero indicates that the well opening is unknown or below the modeled bedrock surface.
\item[frac] transmissivity fraction for a model cell,
where transmissivity is defined as \code{hk} multiplied by \code{thk}.
\item[\dots] pumping rate allocated to the model cell for each time period
specified by \code{rate.col}, in cubic meters per day.
The pumping rate is calculated by multiplying the pumping demand for a well
(specified in \code{wells}) by \code{frac}.

\end{description}

\end{Value}
%
\begin{Author}\relax
J.C. Fisher, U.S. Geological Survey, Idaho Water Science Center

A.H. Wylie, Idaho Department of Water Resources
\end{Author}
%
\begin{Examples}
\begin{ExampleCode}
## Not run: # see Appendix D. Uncalibrated Groundwater-Flow Model

\end{ExampleCode}
\end{Examples}
\inputencoding{utf8}
\HeaderA{hill.shading}{Land Surface Hill Shading}{hill.shading}
\keyword{datasets}{hill.shading}
%
\begin{Description}\relax
Hill shading of the Wood River Valley and surrounding area.
\end{Description}
%
\begin{Usage}
\begin{verbatim}
hill.shading
\end{verbatim}
\end{Usage}
%
\begin{Format}
An object of \code{RasterLayer} class.
Each cell on the surface grid represents the hill shade.
Geographic coordinates are in units of meters, in conformance with the
North American Datum of 1983 (NAD 83), and placed in the
Idaho Transverse Mercator projection (\Rhref{https://www.idwr.idaho.gov/GIS/IDTM/}{IDTM}).
The spatial grid is composed of 3,108 rows and 2,360 columns,
and has cell sizes that are constant at 20 meters by 20 meters.
\end{Format}
%
\begin{Source}\relax
Calculated from the slope and aspect of the \code{\LinkA{land.surface}{land.surface}} dataset
using the \code{terrain} and \code{hillShade} functions;
see the appendix C. Package Dataset Creation for the \R{} code used in this calculation.
\end{Source}
%
\begin{Examples}
\begin{ExampleCode}
raster::image(hill.shading, length(hill.shading), col = grey(0:255 / 255), asp = 1,
              axes = FALSE, xlab = "", ylab = "")

\end{ExampleCode}
\end{Examples}
\inputencoding{utf8}
\HeaderA{idaho}{U.S. State of Idaho}{idaho}
\keyword{datasets}{idaho}
%
\begin{Description}\relax
Boundary of Idaho, a state in the northwestern region of the United States.
\end{Description}
%
\begin{Usage}
\begin{verbatim}
idaho
\end{verbatim}
\end{Usage}
%
\begin{Format}
An object of SpatialPolygons class containing 1 Polygons.
Cartographic boundary at 5m (1:5,000,000) resolution.
Geographic coordinates are in units of meters, in conformance with the
North American Datum of 1983 (NAD 83), and placed in the
Idaho Transverse Mercator projection (\Rhref{https://www.idwr.idaho.gov/GIS/IDTM/}{IDTM}).
\end{Format}
%
\begin{Source}\relax
U.S. Department of Commerce, U.S. Census Bureau,
Geography Division/Cartographic Products Branch.
A simplified representation of the State of Idaho from the 2014 Census Bureau's
MAF/\Rhref{https://www.census.gov/geo/maps-data/data/tiger.html}{TIGER} geographic database.
\end{Source}
%
\begin{Examples}
\begin{ExampleCode}
sp::plot(idaho, col = "#EAE2CF", border = "#BFA76F", lwd = 0.5)
print(idaho)

\end{ExampleCode}
\end{Examples}
\inputencoding{utf8}
\HeaderA{irr.entities}{Irrigation Entities}{irr.entities}
\keyword{datasets}{irr.entities}
%
\begin{Description}\relax
Delineation of areas served by a group of surface-water and (or) groundwater diversions.
\end{Description}
%
\begin{Usage}
\begin{verbatim}
irr.entities
\end{verbatim}
\end{Usage}
%
\begin{Format}
An object of SpatialPolygonsDataFrame class containing 235 Polygons and
a data.frame with the following variables:
\begin{description}

\item[EntityName] name of the irrigation entity served by a group of diversions.
\item[Source] water source, either ``Mixed'' for a mixture of surface water and groundwater,
``SW Only'' for surface-water only, or ``GW Only'' for groundwater only.
\item[EntitySrce] concatenation of the \code{EntityName} and \code{Source} character strings.
\item[PrecipZone] name of the precipitation zone.
See \code{\LinkA{precip.zones}{precip.zones}} dataset for details.

\end{description}

Geographic coordinates are in units of meters, in conformance with the
North American Datum of 1983 (NAD 83), and placed in the
Idaho Transverse Mercator projection (\Rhref{https://www.idwr.idaho.gov/GIS/IDTM/}{IDTM}).
\end{Format}
%
\begin{Source}\relax
Idaho Department of Water Resources (IDWR), accessed on December 11, 2014;
derived from IDWR water rights database, Blaine County tax lot data,
and IDWR irrigated land classification files.
\end{Source}
%
\begin{Examples}
\begin{ExampleCode}
sp::plot(irr.entities)
print(irr.entities)

\end{ExampleCode}
\end{Examples}
\inputencoding{utf8}
\HeaderA{irr.lands}{Irrigated Lands}{irr.lands}
\keyword{datasets}{irr.lands}
%
\begin{Description}\relax
Classification of irrigated lands in the Wood River Valley and surrounding areas;
available for years 1996, 2000, 2002, 2006, 2008, 2009, and 2010.
\end{Description}
%
\begin{Usage}
\begin{verbatim}
irr.lands
\end{verbatim}
\end{Usage}
%
\begin{Format}
An object of list class with 7 SpatialPolygonsDataFrame components.
The data.frame associated with each SpatialPolygons object has the following variable:
\begin{description}

\item[Status] status of land during the year reviewed,
either ``irrigated'', ``semi-irrigated'', or ``non-irrigated''.

\end{description}

Geographic coordinates are in units of meters, in conformance with the
North American Datum of 1983 (NAD 83), and placed in the
Idaho Transverse Mercator projection (\Rhref{https://www.idwr.idaho.gov/GIS/IDTM/}{IDTM}).
\end{Format}
%
\begin{Source}\relax
Idaho Department of Water Resources, accessed on November 17, 2014;
polygons derived from U.S. Department of Agriculture Common Land Unit polygons
with some refinement of polygons.
Irrigation status interpreted using satellite imagery and aerial photography.
\end{Source}
%
\begin{SeeAlso}\relax
\code{\LinkA{irr.lands.year}{irr.lands.year}}
\end{SeeAlso}
%
\begin{Examples}
\begin{ExampleCode}
sp::spplot(irr.lands[["2010"]], "Status")
print(irr.lands)

\end{ExampleCode}
\end{Examples}
\inputencoding{utf8}
\HeaderA{irr.lands.year}{Irrigation Lands for a Given Year}{irr.lands.year}
\keyword{datasets}{irr.lands.year}
%
\begin{Description}\relax
Annual land classification for irrigation practices is only available for select years.
For missing years, this dataset provides substitute years when land-classification was
available (see \code{\LinkA{irr.lands}{irr.lands}}).
\end{Description}
%
\begin{Usage}
\begin{verbatim}
irr.lands.year
\end{verbatim}
\end{Usage}
%
\begin{Format}
An object of data.frame class with 16 records and the following variables:
\begin{description}

\item[Year] year with a required date format of YYYY.
\item[IL\_Year] substitute year with a required date format of YYYY.

\end{description}
\end{Format}
%
\begin{Source}\relax
Idaho Department of Water Resources, accessed on April 25, 2014
\end{Source}
%
\begin{Examples}
\begin{ExampleCode}
str(irr.lands.year)

\end{ExampleCode}
\end{Examples}
\inputencoding{utf8}
\HeaderA{kriging.zones}{Kriging Zones}{kriging.zones}
\keyword{datasets}{kriging.zones}
%
\begin{Description}\relax
Location of kriging zones in the Wood River Valley, used in parameter estimation.
\end{Description}
%
\begin{Usage}
\begin{verbatim}
kriging.zones
\end{verbatim}
\end{Usage}
%
\begin{Format}
An object of SpatialPolygonsDataFrame class containing 18 Polygons and a
data.frame with the following variables:
\begin{description}

\item[Zone] kriging zone, interpolation in this zone is based on the
parameter values of pilot points located within this zone.
\item[Layer] model layer
\item[Name] local name

\end{description}

Geographic coordinates are in units of meters, in conformance with the
North American Datum of 1983 (NAD 83), and placed in the
Idaho Transverse Mercator projection (\Rhref{https://www.idwr.idaho.gov/GIS/IDTM/}{IDTM}).
\end{Format}
%
\begin{Source}\relax
Idaho Department of Water Resources, accessed on June 11, 2015
\end{Source}
%
\begin{SeeAlso}\relax
\code{\LinkA{pilot.points}{pilot.points}}
\end{SeeAlso}
%
\begin{Examples}
\begin{ExampleCode}
sp::plot(kriging.zones)
str(kriging.zones@data)

\end{ExampleCode}
\end{Examples}
\inputencoding{utf8}
\HeaderA{lakes}{Lakes and Reservoirs}{lakes}
\keyword{datasets}{lakes}
%
\begin{Description}\relax
Lakes and reservoirs of the Wood River Valley and surrounding areas.
\end{Description}
%
\begin{Usage}
\begin{verbatim}
lakes
\end{verbatim}
\end{Usage}
%
\begin{Format}
An object of SpatialPolygonsDataFrame class containing 55 Polygons.
Geographic coordinates are in units of meters, in conformance with the
North American Datum of 1983 (NAD 83), and placed in the
Idaho Transverse Mercator projection (\Rhref{https://www.idwr.idaho.gov/GIS/IDTM/}{IDTM}).
\end{Format}
%
\begin{Source}\relax
Idaho Department of Water Resources
(\Rhref{https://research.idwr.idaho.gov/index.html#GIS-Data}{IDWR}),
accessed on April 2, 2014
\end{Source}
%
\begin{Examples}
\begin{ExampleCode}
sp::plot(lakes, col = "#CCFFFF", border = "#3399CC", lwd = 0.5)
str(lakes@data)

\end{ExampleCode}
\end{Examples}
\inputencoding{utf8}
\HeaderA{land.surface}{Topography of Land Surface}{land.surface}
\keyword{datasets}{land.surface}
%
\begin{Description}\relax
The Wood River Valley (WRV) is a geologic feature located in south-central Idaho.
This dataset gives the topography of the land surface in the WRV and vicinity.
\end{Description}
%
\begin{Usage}
\begin{verbatim}
land.surface
\end{verbatim}
\end{Usage}
%
\begin{Format}
An object of SpatialGridDataFrame class.
Each cell on the surface grid represents an elevation in meters above the
North American Vertical Datum of 1988 (NAVD 88).
Geographic coordinates are in units of meters, in conformance with the
North American Datum of 1983 (NAD 83), and placed in the
Idaho Transverse Mercator projection (\Rhref{https://www.idwr.idaho.gov/GIS/IDTM/}{IDTM}).
The spatial grid is composed of 565 rows and 429 columns,
and has cell sizes that are constant at 100 meters by 100 meters.
\end{Format}
%
\begin{Source}\relax
The National Map (\Rhref{https://nationalmap.gov/elevation.html}{TNM})
1/3-arc-second raster (Gesch, 2007; Gesch and others, 2002),
accessed on December 1, 2015.
This dataset can be downloaded in a Esri ArcGRID format using the
\Rhref{https://viewer.nationalmap.gov/viewer/}{The National Map Viewer}.
Elevation datasets are distributed in geographic coordinates in units of decimal degrees,
and in conformance with the NAD 83.
Elevation values are in meters above the NAVD 88.
The west, east, south, and north bounding coordinates for this dataset are
-115, -114, 43, and 44 decimal degrees, respectively.
Post-processing includes:
(1) project the values of the elevation dataset into the \code{\LinkA{alluvium.thickness}{alluvium.thickness}}
spatial grid using bilinear interpolation, and
(2) set values in cells where the elevation of the alluvium bottom is missing to NA.
\end{Source}
%
\begin{References}\relax
Gesch, D.B., 2007, The National Elevation Dataset, in Maune, D., ed.,
Digital Elevation Model Technologies and Applications: The DEM Users Manual,
2nd Edition: Bethesda, Maryland, American Society for Photogrammetry and Remote Sensing,
p. 99-118.

Gesch, D., Oimoen, M., Greenlee, S., Nelson, C., Steuck, M., and Tyler, D., 2002,
The National Elevation Dataset: Photogrammetric Engineering and Remote Sensing,
v. 68, no. 1, p. 5-11.
\end{References}
%
\begin{Examples}
\begin{ExampleCode}
raster::image(land.surface)
summary(land.surface)

\end{ExampleCode}
\end{Examples}
\inputencoding{utf8}
\HeaderA{major.roads}{Major Roads}{major.roads}
\keyword{datasets}{major.roads}
%
\begin{Description}\relax
Major roads in the Wood River Valley and surrounding areas.
\end{Description}
%
\begin{Usage}
\begin{verbatim}
major.roads
\end{verbatim}
\end{Usage}
%
\begin{Format}
An object of SpatialLinesDataFrame class containing 475 Lines.
Geographic coordinates are in units of meters, in conformance with the
North American Datum of 1983 (NAD 83), and placed in the
Idaho Transverse Mercator projection (\Rhref{https://www.idwr.idaho.gov/GIS/IDTM/}{IDTM}).
\end{Format}
%
\begin{Source}\relax
Idaho Department of Water Resources
(\Rhref{https://research.idwr.idaho.gov/index.html#GIS-Data}{IDWR}),
accessed on October 20, 2015
\end{Source}
%
\begin{Examples}
\begin{ExampleCode}
sp::plot(major.roads)
str(major.roads@data)

\end{ExampleCode}
\end{Examples}
\inputencoding{utf8}
\HeaderA{map.labels}{Map Labels}{map.labels}
\keyword{datasets}{map.labels}
%
\begin{Description}\relax
Map labels in the Wood River Valley and surrounding areas.
\end{Description}
%
\begin{Usage}
\begin{verbatim}
map.labels
\end{verbatim}
\end{Usage}
%
\begin{Format}
An object of SpatialPointsDataFrame class containing
40 points with the following variables:
\begin{description}

\item[label] text to be written.
\item[cex] character expansion factor
\item[col,font] color and font to be used, respectively.
\item[srt] string rotation in degrees.

\end{description}

Geographic coordinates are in units of meters, in conformance with the
North American Datum of 1983 (NAD 83), and placed in the
Idaho Transverse Mercator projection (\Rhref{https://www.idwr.idaho.gov/GIS/IDTM/}{IDTM}).
\end{Format}
%
\begin{Source}\relax
Best estimates of map label locations.
\end{Source}
%
\begin{Examples}
\begin{ExampleCode}
sp::plot(map.labels, col = "red")
lab <- cbind(map.labels@coords, map.labels@data)
for (i in seq_len(nrow(lab))) {
  text(lab$x[i], lab$y[i], labels = lab$label[i], cex = lab$cex[i],
       col = lab$col[i], font = lab$font[i], srt = lab$srt[i])
}

\end{ExampleCode}
\end{Examples}
\inputencoding{utf8}
\HeaderA{misc.locations}{Miscellaneous Locations}{misc.locations}
\keyword{datasets}{misc.locations}
%
\begin{Description}\relax
Miscellaneous locations in the Bellevue triangle area.
\end{Description}
%
\begin{Usage}
\begin{verbatim}
misc.locations
\end{verbatim}
\end{Usage}
%
\begin{Format}
An object of SpatialPointsDataFrame class containing 3 points
with the following variable:
\begin{description}

\item[label] description of point location.

\end{description}

Geographic coordinates are in units of meters, in conformance with the
North American Datum of 1983 (NAD 83), and placed in the
Idaho Transverse Mercator projection (\Rhref{https://www.idwr.idaho.gov/GIS/IDTM/}{IDTM}).
\end{Format}
%
\begin{Source}\relax
Idaho Department of Water Resources
(\Rhref{https://research.idwr.idaho.gov/index.html#GIS-Data}{IDWR}),
accessed on December 23, 2015
\end{Source}
%
\begin{Examples}
\begin{ExampleCode}
sp::plot(misc.locations, pch = 20, col = "red")
text(misc.locations, labels = misc.locations@data$label, pos = 3, cex = 0.6)

\end{ExampleCode}
\end{Examples}
\inputencoding{utf8}
\HeaderA{misc.seepage}{Direct Recharge from Miscellaneous Seepage Sites}{misc.seepage}
\keyword{datasets}{misc.seepage}
%
\begin{Description}\relax
Direct recharge from miscellaneous seepage sites in the Wood River Valley, Idaho.
\end{Description}
%
\begin{Usage}
\begin{verbatim}
misc.seepage
\end{verbatim}
\end{Usage}
%
\begin{Format}
An object of data.frame class with 2 records and the following variables:
\begin{description}

\item[RechSite] name of the recharge site, see \code{\LinkA{bellevue.wwtp.ponds}{bellevue.wwtp.ponds}} and
\code{\LinkA{bypass.canal}{bypass.canal}} datasets.
\item[199501,\dots,201012] monthly volume of recharge during a stress period, in cubic meters.
The variable name is specified as year and month.

\end{description}
\end{Format}
%
\begin{Source}\relax
Idaho Department of Water Resources, accessed on January 15, 2015
\end{Source}
%
\begin{Examples}
\begin{ExampleCode}
str(misc.seepage)

\end{ExampleCode}
\end{Examples}
\inputencoding{utf8}
\HeaderA{obs.wells}{Observation Wells}{obs.wells}
\keyword{datasets}{obs.wells}
%
\begin{Description}\relax
Observation wells in the Wood River Valley aquifer system.
\end{Description}
%
\begin{Usage}
\begin{verbatim}
obs.wells
\end{verbatim}
\end{Usage}
%
\begin{Format}
An object of SpatialPointsDataFrame class containing 776 points
with the following variables:
\begin{description}

\item[id] unique well identifier used in this study.
\item[SiteNo] unique well identifier within the
National Water Information System (NWIS).
\item[SITEIDIDWR] unique well identifier within the
Idaho Department of Water Resources (IDWR) hydrologic database.
\item[WELLNUMBER] USGS or IDWR site name for the well.
\item[PESTNAME] unique well identifier for PEST.
\item[METHODDRIL] drilling method
\item[TOTALDEPTH] depth at which drilling stopped, in feet.
\item[OPENINGMIN] top of the screened interval, in feet.
\item[OPENINGMAX] bottom of the screened interval, in feet.
\item[COMPLETION] date on which the well drilling and construction stopped.
\item[WCWELLID] well construction well identifier.
\item[ALTITUDE] land surface elevation, in feet.
\item[ALTMETHOD] method for obtaining the land surface elevation.
\item[XYMETHOD] method of obtaining the spatial coordinates.
\item[BASINNO] basin number
\item[COUNTYNAME] Idaho county name
\item[TWPRGE] township and range the well is located in.
\item[SITENAME] local name for well.
\item[desc] description of well type.
\item[TopOpen1] depth to the top of the first open interval in a groundwater well,
in meters below land surface.
\item[BotOpen1] depth to the bottom of the first open interval in a groundwater well,
in meters below land surface.
\item[TopOpen2] not applicable
\item[BotOpen2] not applicable

\end{description}
\end{Format}
%
\begin{Source}\relax
Idaho Department of Water Resources well construction database,
accessed on June 29, 2015
\end{Source}
%
\begin{SeeAlso}\relax
\code{\LinkA{obs.wells.head}{obs.wells.head}}
\end{SeeAlso}
%
\begin{Examples}
\begin{ExampleCode}
sp::plot(obs.wells)
str(obs.wells@data)

\end{ExampleCode}
\end{Examples}
\inputencoding{utf8}
\HeaderA{obs.wells.head}{Hydraulic Heads in Observation Wells}{obs.wells.head}
\keyword{datasets}{obs.wells.head}
%
\begin{Description}\relax
Hydraulic-head (groundwater-level) measurements recorded in observation wells in the
Wood River Valley aquifer system.
Values are used as observations during the parameter estimation process.
\end{Description}
%
\begin{Usage}
\begin{verbatim}
obs.wells.head
\end{verbatim}
\end{Usage}
%
\begin{Format}
An object of class data.frame with 3,477 records and the following variables:
\begin{description}

\item[PESTNAME] unique well identifier for PEST.
\item[DateTime] date and time when the measurement was recorded.
\item[Head] groundwater-level measurement (hydraulic head)
in meters above the North American Vertical Datum of 1988 (NAVD 88).

\end{description}
\end{Format}
%
\begin{Source}\relax
Idaho Department of Water Resources, accessed on June 26, 2015
\end{Source}
%
\begin{SeeAlso}\relax
\code{\LinkA{obs.wells}{obs.wells}}
\end{SeeAlso}
%
\begin{Examples}
\begin{ExampleCode}
str(obs.wells.head)

\end{ExampleCode}
\end{Examples}
\inputencoding{utf8}
\HeaderA{pilot.points}{Pilot Points}{pilot.points}
\keyword{datasets}{pilot.points}
%
\begin{Description}\relax
Location of pilot points in the model domain.
\end{Description}
%
\begin{Usage}
\begin{verbatim}
pilot.points
\end{verbatim}
\end{Usage}
%
\begin{Format}
An object of SpatialPointsDataFrame class containing 106 points
with the following variables:
\begin{description}

\item[Layer] model layer
\item[Zone] kriging-zone identifier

\end{description}

Geographic coordinates are in units of meters, in conformance with the
North American Datum of 1983 (NAD 83), and placed in the
Idaho Transverse Mercator projection (\Rhref{https://www.idwr.idaho.gov/GIS/IDTM/}{IDTM}).
\end{Format}
%
\begin{Source}\relax
Idaho Department of Water Resources, accessed on June 11, 2015
\end{Source}
%
\begin{SeeAlso}\relax
\code{\LinkA{kriging.zones}{kriging.zones}}
\end{SeeAlso}
%
\begin{Examples}
\begin{ExampleCode}
sp::plot(pilot.points)
str(pilot.points@data)

\end{ExampleCode}
\end{Examples}
\inputencoding{utf8}
\HeaderA{pod.gw}{Points of Diversion for Groundwater}{pod.gw}
\keyword{datasets}{pod.gw}
%
\begin{Description}\relax
Points of diversion for groundwater within the Wood River Valley model study area.
\end{Description}
%
\begin{Usage}
\begin{verbatim}
pod.gw
\end{verbatim}
\end{Usage}
%
\begin{Format}
An object of class data.frame with 1,755 records and the following variables:
\begin{description}

\item[WMISNumber] unique number assigned to a water right point of diversion.
\item[WaterRight] number identifying a specific authorization to use
water in a prescribed manner.
\item[EntityName] name of the irrigation entity the point of diversion is assigned to.
\item[EntitySrce] source of water for an irrigation entity.
Possible sources of water include surface water, groundwater and mixed.
Mixed source entities derive water from both groundwater and surface water.
\item[Pdate] priority date, the date the water right was established.
\item[IrrRate] irrigation rate in cubic meters per day,
the maximum permitted water use rate associated with a water right.

\end{description}
\end{Format}
%
\begin{Source}\relax
Idaho Department of Water Resources water rights database,
accessed on December 11, 2014
\end{Source}
%
\begin{SeeAlso}\relax
\code{\LinkA{pod.wells}{pod.wells}}
\end{SeeAlso}
%
\begin{Examples}
\begin{ExampleCode}
summary(pod.gw)

\end{ExampleCode}
\end{Examples}
\inputencoding{utf8}
\HeaderA{pod.wells}{Well Completions}{pod.wells}
\keyword{datasets}{pod.wells}
%
\begin{Description}\relax
Well completions for pumping wells in the Wood River Valley aquifer system.
\end{Description}
%
\begin{Usage}
\begin{verbatim}
pod.wells
\end{verbatim}
\end{Usage}
%
\begin{Format}
An object of SpatialPointsDataFrame class containing 1,243 points
with the following variables:
\begin{description}

\item[WMISNumber] is a unique number assigned to a water right point of diversion.
\item[WellUse] permitted use(s) for a groundwater well.
\item[TopOpen1] depth to the top of the first open interval in a groundwater well,
in meters below land surface.
\item[BotOpen1] depth to the bottom of the first open interval in a groundwater well,
in meters below land surface.
\item[TopOpen2] depth to the top of the second open interval in a groundwater well,
in meters below land surface.
\item[BotOpen2] depth to the bottom of the second open interval in a groundwater well,
in meters below land surface.

\end{description}

Geographic coordinates are in units of meters, in conformance with the
North American Datum of 1983 (NAD 83), and placed in the
Idaho Transverse Mercator projection (\Rhref{https://www.idwr.idaho.gov/GIS/IDTM/}{IDTM}).
\end{Format}
%
\begin{Source}\relax
Idaho Department of Water Resources water rights database,
accessed on November 29, 2014
\end{Source}
%
\begin{SeeAlso}\relax
\code{\LinkA{pod.gw}{pod.gw}}
\end{SeeAlso}
%
\begin{Examples}
\begin{ExampleCode}
sp::plot(pod.wells)
str(pod.wells@data)

\end{ExampleCode}
\end{Examples}
\inputencoding{utf8}
\HeaderA{precip.zones}{Precipitation Zones}{precip.zones}
\keyword{datasets}{precip.zones}
%
\begin{Description}\relax
Precipitation zones specified for the Wood River Valley and surrounding areas.
There are three precipitation zones, each containing a single weather station.
Precipitation zones were distributed to maintain the geographic similarity between
weather stations and zones.
\end{Description}
%
\begin{Usage}
\begin{verbatim}
precip.zones
\end{verbatim}
\end{Usage}
%
\begin{Format}
An object of SpatialPolygonsDataFrame class containing 3 Polygons and a
data.frame with the following variables:
\begin{description}

\item[ID] numeric identifier assigned to the polygon.
\item[PrecipZone] name of the precipitation zone:
``Ketchum'', the northernmost zone with data from the
Ketchum National Weather Service coop weather station.
``Hailey'', central zone with data from the
Hailey 3NNW National Weather Service coop weather station.
``Picabo'', southernmost zone with data from the
Picabo AgriMet weather station.

\end{description}

Geographic coordinates are in units of meters, in conformance with the
North American Datum of 1983 (NAD 83), and placed in the
Idaho Transverse Mercator projection (\Rhref{https://www.idwr.idaho.gov/GIS/IDTM/}{IDTM}).
\end{Format}
%
\begin{Source}\relax
Created using the northing midpoint between weather stations,
see \code{\LinkA{weather.stations}{weather.stations}} dataset.
\end{Source}
%
\begin{SeeAlso}\relax
\code{\LinkA{precipitation}{precipitation}}
\end{SeeAlso}
%
\begin{Examples}
\begin{ExampleCode}
col <- c("#D1F2A5", "#FFC48C", "#F56991")
sp::plot(precip.zones, col = col)
legend("topright", legend = precip.zones@data$PrecipZone, fill = col, bty = "n")
sp::plot(alluvium.extent, add = TRUE)

print(precip.zones)

\end{ExampleCode}
\end{Examples}
\inputencoding{utf8}
\HeaderA{precipitation}{Precipitation Rate}{precipitation}
\keyword{datasets}{precipitation}
%
\begin{Description}\relax
Precipitation rates in the Wood River Valley and surrounding areas.
\end{Description}
%
\begin{Usage}
\begin{verbatim}
precipitation
\end{verbatim}
\end{Usage}
%
\begin{Format}
An object of data.frame class with 582 records and the following variables:
with the following variables:
\begin{description}

\item[YearMonth] year and month during which precipitation were recorded,
with a required date format of \code{YYYYMM}.
\item[PrecipZone] name of the precipitation zone,
see \code{\LinkA{precip.zones}{precip.zones}} dataset for details.
\item[Precip] monthly depth of precipitation accounting for spring melt, in meters.
\item[Precip.raw] monthly depth of precipitation recorded at the weather station,
in meters.

\end{description}
\end{Format}
%
\begin{Source}\relax
Idaho Department of Water Resources, accessed on April 24, 2015
\end{Source}
%
\begin{References}\relax
National Oceanic and Atmospheric Administration's National Weather Service
(\Rhref{https://www.ncdc.noaa.gov/data-access/land-based-station-data/land-based-datasets/cooperative-observer-network-coop}{NWS}) Cooperative Observer Program

U.S. Bureau of Reclamation's Cooperative Agricultural Weather Network
(\Rhref{https://www.usbr.gov/pn/agrimet/}{AgriMet})
\end{References}
%
\begin{SeeAlso}\relax
\code{\LinkA{precip.zones}{precip.zones}}, \code{\LinkA{swe}{swe}}
\end{SeeAlso}
%
\begin{Examples}
\begin{ExampleCode}
str(precipitation)

d <- precipitation
d <- data.frame(Date = as.Date(paste0(d$YearMonth, "15"), format = "\%Y\%m\%d"),
                Precip = d$Precip)
zones <- c("Hailey", "Ketchum", "Picabo")
d1 <- d[precipitation$PrecipZone == zones[1], ]
d2 <- d[precipitation$PrecipZone == zones[2], ]
d3 <- d[precipitation$PrecipZone == zones[3], ]
d <- merge(merge(d1, d2, by = "Date"), d3, by = "Date")

col <- c("red", "blue", "green")
ylab <- paste("Precipitation in", c("meters", "feet"))
inlmisc::PlotGraph(d, ylab = ylab, col = col, lty = 1:3, conversion.factor = 3.28084)
legend("topright", zones, col = col, lty = 1:3, inset = 0.02, cex = 0.7,
       box.lty = 1, bg = "#FFFFFFE7")

graphics.off()

\end{ExampleCode}
\end{Examples}
\inputencoding{utf8}
\HeaderA{priority.cuts}{Priority Cuts}{priority.cuts}
\keyword{datasets}{priority.cuts}
%
\begin{Description}\relax
Priority cut dates applied to Big Wood River above Magic Reservoir and
Silver Creek by Water District 37 and 37M at the end of each month.
\end{Description}
%
\begin{Usage}
\begin{verbatim}
priority.cuts
\end{verbatim}
\end{Usage}
%
\begin{Format}
An object of data.frame class with 112 records
and the following variables:
\begin{description}

\item[YearMonth] year and month during of the priority cut date,
with a required date format of YYYYMM.
\item[Pdate\_BWR] date of the priority cut applied to
Big Wood River above Magic Reservoir by Water District 37.
\item[Pdate\_SC] date of the priority cut applied to
Silver Creek by Water District 37M.

\end{description}
\end{Format}
%
\begin{Source}\relax
Idaho Department of Water Resources, accessed on November 17, 2014;
compiled priority cut dates in effect at the end of each month,
derived from Water District 37 and 37M records.
\end{Source}
%
\begin{Examples}
\begin{ExampleCode}
str(priority.cuts)

\end{ExampleCode}
\end{Examples}
\inputencoding{utf8}
\HeaderA{public.parcels}{Public Land Parcels}{public.parcels}
\keyword{datasets}{public.parcels}
%
\begin{Description}\relax
Non-irrigated public land parcels in the Wood River Valley and surrounding areas.
\end{Description}
%
\begin{Usage}
\begin{verbatim}
public.parcels
\end{verbatim}
\end{Usage}
%
\begin{Format}
An object of SpatialPolygons class containing 669 Polygons.
Geographic coordinates are in units of meters, in conformance with the
North American Datum of 1983 (NAD 83), and placed in the
Idaho Transverse Mercator projection (\Rhref{https://www.idwr.idaho.gov/GIS/IDTM/}{IDTM}).
\end{Format}
%
\begin{Source}\relax
Idaho Department of Water Resources, accessed on November 29, 2014;
derived from Blaine County tax lots and aerial photography.
\end{Source}
%
\begin{Examples}
\begin{ExampleCode}
sp::plot(public.parcels)
print(public.parcels)

\end{ExampleCode}
\end{Examples}
\inputencoding{utf8}
\HeaderA{r.canals}{Rasterized Canals}{r.canals}
\keyword{datasets}{r.canals}
%
\begin{Description}\relax
Canal systems of the Wood River Valley and surrounding areas transferred to raster cells.
\end{Description}
%
\begin{Usage}
\begin{verbatim}
r.canals
\end{verbatim}
\end{Usage}
%
\begin{Format}
An object of RasterLayer class with indexed cell values linked to
a raster attribute table (RAT).
The RAT is a \code{data.frame} with the following components:
\begin{description}

\item[ID] cell index
\item[COUNT] frequency of the cell index in the raster grid.
\item[EntityName] name of the irrigation entity served by the canal system.

\end{description}
\end{Format}
%
\begin{Source}\relax
Calculated by transferring the \code{\LinkA{canals}{canals}} dataset to grid cells in the
\code{\LinkA{land.surface}{land.surface}} dataset using the \code{rasterize} function;
see the appendix C. Package Dataset Creation for the \R{} code used in this calculation.
\end{Source}
%
\begin{Examples}
\begin{ExampleCode}
sp::plot(r.canals)
print(levels(r.canals)[[1]])

\end{ExampleCode}
\end{Examples}
\inputencoding{utf8}
\HeaderA{reach.recharge}{Stream-Aquifer Flow Exchange Along River Reaches}{reach.recharge}
\keyword{datasets}{reach.recharge}
%
\begin{Description}\relax
Stream-aquifer flow exchange along river reaches specified as aquifer recharge.
Values used as observations in parameter estimation.
\end{Description}
%
\begin{Usage}
\begin{verbatim}
reach.recharge
\end{verbatim}
\end{Usage}
%
\begin{Format}
An object of \code{data.frame} class with 192 records and the following variables:
\begin{description}

\item[YearMonth] year and month of the measurement record,
with a required date format of \code{YYYYMM}.
\item[nKet\_Hai] stream-aquifer flow exchange in the Big Wood River,
near Ketchum to Hailey river reach, in cubic meters per day.
\item[Hai\_StC] stream-aquifer flow exchange in the Big Wood River,
Hailey to Stanton Crossing river reach, in cubic meters per day.
\item[WillowCr] stream-aquifer flow exchange in the Willow Creek river reach,
in cubic meters per day.
\item[SilverAbv] stream-aquifer flow exchange in Silver Creek,
above Sportsman Access river reach, in cubic meters per day.
\item[SilverBlw] stream-aquifer flow exchange in Silver Creek,
Sportsman Access to near Picabo river reach, in cubic meters per day.

\end{description}
\end{Format}
%
\begin{Details}\relax
A positive stream-aquifer flow exchange indicates aquifer recharge
(a losing river reach).
\end{Details}
%
\begin{Source}\relax
Calculated from continuous stream flow measurements, diversion data,
return flow data, and exchange well data using a flow difference method to
estimate groundwater inflows and outflows along a river reach,
accessed on September 1, 2015.
Derived from U.S. Geological Survey, Idaho Power Company,
and Water District 37 and 37M records.
\end{Source}
%
\begin{Examples}
\begin{ExampleCode}
str(reach.recharge)

\end{ExampleCode}
\end{Examples}
\inputencoding{utf8}
\HeaderA{river.reaches}{Major River Reaches}{river.reaches}
\keyword{datasets}{river.reaches}
%
\begin{Description}\relax
Major river reaches of the Wood River Valley, Idaho.
\end{Description}
%
\begin{Usage}
\begin{verbatim}
river.reaches
\end{verbatim}
\end{Usage}
%
\begin{Format}
An object of SpatialLinesDataFrame class containing 22 Lines and a
data.frame with the following variables:
\begin{description}

\item[Reach] name of the subreaches measured in U.S. Geological Survey (USGS)
seepage survey.
\item[BigReach] name of the reaches for which time series targets are available for
part or all of the calibration period.
\item[DrainRiver] model boundary assignment, either ``drain'' or ``river''.
\item[RchAvg] estimated average reach gain in cubic meters per day for 1995-2010
based on a combination of gage data and the USGS seepage survey.
\item[BigRAv] estimated average reach gain in cubic meters per day for 1995-2010
based on gage data.
\item[ReachNo] reach number identifier.
\item[Depth] estimated average depth in meters of water in reach,
measured from the air-water interface to the top of the riverbed sediments.
\item[BedThk] estimated thickness in meters of the saturated riverbed sediments.

\end{description}

Geographic coordinates are in units of meters, in conformance with the
North American Datum of 1983 (NAD 83), and placed in the
Idaho Transverse Mercator projection (\Rhref{https://www.idwr.idaho.gov/GIS/IDTM/}{IDTM}).
\end{Format}
%
\begin{Source}\relax
Idaho Department of Water Resources, accessed on June 6, 2015
\end{Source}
%
\begin{Examples}
\begin{ExampleCode}
sp::plot(river.reaches)
str(river.reaches@data)

\end{ExampleCode}
\end{Examples}
\inputencoding{utf8}
\HeaderA{rs.entities}{Rasterized Monthly Irrigation Entities}{rs.entities}
\keyword{datasets}{rs.entities}
%
\begin{Description}\relax
Irrigation entities of the Wood River Valley and surrounding areas transferred to raster cells.
\end{Description}
%
\begin{Usage}
\begin{verbatim}
rs.entities
\end{verbatim}
\end{Usage}
%
\begin{Format}
An object of RasterStack class containing a 192 RasterLayer objects,
one layer for each month in the 1995-2010 time period.
Geographic coordinates are in units of meters, in conformance with the
North American Datum of 1983 (NAD 83), and placed in the
Idaho Transverse Mercator projection (\Rhref{https://www.idwr.idaho.gov/GIS/IDTM/}{IDTM}).
For each raster layer, indexed cell values are linked to a raster attribute table (RAT).
The RAT is an object of data.frame class with the following components:
\begin{description}

\item[ID] cell index
\item[COUNT] frequency of the cell index in the raster grid.
\item[EntityName] name of the irrigation entity served by a group of diversions.

\end{description}
\end{Format}
%
\begin{Source}\relax
Calculated by transferring the \code{\LinkA{entity.components}{entity.components}} dataset to grid cells
in the \code{\LinkA{land.surface}{land.surface}} dataset using the rasterize function;
see the appendix C. Package Dataset Creation for the \R{} code used in this calculation.
\end{Source}
%
\begin{Examples}
\begin{ExampleCode}
names(rs.entities)
sp::plot(rs.entities[["199507"]])
print(levels(rs.entities[["199507"]])[[1]])

\end{ExampleCode}
\end{Examples}
\inputencoding{utf8}
\HeaderA{rs.rech.non.irr}{Rasterized Monthly Recharge on Non-Irrigated Lands}{rs.rech.non.irr}
\keyword{datasets}{rs.rech.non.irr}
%
\begin{Description}\relax
Aerial recharge and discharge on non-irrigated lands of the Wood River Valley and
surrounding areas transferred to raster cells.
\end{Description}
%
\begin{Usage}
\begin{verbatim}
rs.rech.non.irr
\end{verbatim}
\end{Usage}
%
\begin{Format}
An object of RasterStack class containing 192 RasterLayer objects,
one layer for each month in the 1995-2010 time period.
Each cell on a layers surface grid represents the monthly recharge in meters.
Geographic coordinates are in units of meters, in conformance with the
North American Datum of 1983 (NAD 83), and placed in the
Idaho Transverse Mercator projection (\Rhref{https://www.idwr.idaho.gov/GIS/IDTM/}{IDTM}).
\end{Format}
%
\begin{Source}\relax
Calculated from the \code{\LinkA{et}{et}}, \code{\LinkA{precipitation}{precipitation}},
\code{\LinkA{precip.zones}{precip.zones}}, and \code{\LinkA{soils}{soils}} datasets;
see the appendix C. Package Dataset Creation for the \R{} code used in this calculation.
\end{Source}
%
\begin{Examples}
\begin{ExampleCode}
names(rs.rech.non.irr)
sp::plot(rs.rech.non.irr[["199507"]])

\end{ExampleCode}
\end{Examples}
\inputencoding{utf8}
\HeaderA{RunWaterBalance}{Run Water Balance}{RunWaterBalance}
\keyword{manip}{RunWaterBalance}
%
\begin{Description}\relax
This function estimates natural and incidental groundwater recharge at the water table,
and pumping demand at production wells.
A water-balance approach is used to calculate these volumetric flow rate estimates,
where positive values are flow into the aqufer system (groundwater recharge),
and negative values are flow out of the system (groundwater discharge).
\end{Description}
%
\begin{Usage}
\begin{verbatim}
RunWaterBalance(r.grid, tr.stress.periods, ss.stress.periods = NULL,
  canal.seep = wrv::canal.seep, comb.sw.irr = wrv::comb.sw.irr,
  div.gw = wrv::div.gw, div.sw = wrv::div.sw, div.ww = wrv::div.ww,
  efficiency = wrv::efficiency, entity.components = wrv::entity.components,
  et = wrv::et, irr.entities = wrv::irr.entities,
  land.surface = wrv::land.surface, pod.gw = wrv::pod.gw,
  priority.cuts = wrv::priority.cuts, r.canals = wrv::r.canals,
  rs.entities = wrv::rs.entities, rs.rech.non.irr = wrv::rs.rech.non.irr,
  verbose = FALSE)
\end{verbatim}
\end{Usage}
%
\begin{Arguments}
\begin{ldescription}
\item[\code{r.grid}] RasterLayer.
Gridded numeric values where NA indicates an `inactive' cell in
the top model layer.

\item[\code{tr.stress.periods}] Date.
Vector of start and end dates for each stress period in the simulation.

\item[\code{ss.stress.periods}] Date.
Vector of start and end dates for stress periods used to create steady-state conditions.

\item[\code{canal.seep}] data.frame.
See \code{\LinkA{canal.seep}{canal.seep}} dataset for details.

\item[\code{comb.sw.irr}] data.frame.
See \code{\LinkA{comb.sw.irr}{comb.sw.irr}} dataset for details.

\item[\code{div.gw}] data.frame.
See \code{\LinkA{div.gw}{div.gw}} dataset for details.

\item[\code{div.sw}] data.frame.
See \code{\LinkA{div.sw}{div.sw}} dataset for details.

\item[\code{div.ww}] data.frame.
See \code{\LinkA{div.gw}{div.gw}} dataset for details.

\item[\code{efficiency}] data.frame.
See \code{\LinkA{efficiency}{efficiency}} dataset for details.

\item[\code{entity.components}] list.
See \code{\LinkA{entity.components}{entity.components}} dataset for details.

\item[\code{et}] RasterStack.
See \code{\LinkA{et}{et}} dataset for details.

\item[\code{irr.entities}] SpatialPolygonsDataFrame.
See \code{\LinkA{irr.entities}{irr.entities}} dataset for details.

\item[\code{land.surface}] RasterLayer.
See \code{\LinkA{land.surface}{land.surface}} dataset for details.

\item[\code{pod.gw}] data.frame.
See \code{\LinkA{pod.gw}{pod.gw}} dataset for details.

\item[\code{priority.cuts}] data.frame.
See \code{\LinkA{priority.cuts}{priority.cuts}} dataset for details.

\item[\code{r.canals}] RasterLayer.
See \code{\LinkA{r.canals}{r.canals}} dataset for details.

\item[\code{rs.entities}] RasterStack.
See \code{\LinkA{rs.entities}{rs.entities}} dataset for details.

\item[\code{rs.rech.non.irr}] RasterStack.
See \code{\LinkA{rs.rech.non.irr}{rs.rech.non.irr}} dataset for details.

\item[\code{verbose}] logical.
Indicates whether to return summary tables:
\code{natural.rech}, \code{inciden.rech}, and \code{pumping.rech}.
\end{ldescription}
\end{Arguments}
%
\begin{Value}
Returns an object of class list with the following components:

(1) Water-table flow data (combines natural and incidental groundwater recharge)
are stored in \code{areal.rech},
an object of RasterStack class with raster layers for each model stress period;
cell values are specified as volumetric flow rates in cubic meters per day.

(2) Production well pumping data are stored in \code{pod.rech},
an object of \code{data.frame} class with the following components:
\begin{description}

\item[WMISNumber] unique number assigned to a water right point of diversion.
\item[ss,199501,\dots,201012] volumetric flow rate, specified for each stress period,
in cubic meters per day.

\end{description}


(3) Natural groundwater recharge data are stored in \code{natural.rech},
an object of data.frame class with the following components:
\begin{description}

\item[YearMonth] calendar year and month YYYYMM.
\item[Area] land-surface area of non-irrigated lands, in square meters.
\item[ET] evapotranspiration on non-irrigated lands, in cubic meters per month.
\item[Rech] volumetric flow rate, in cubic meters per month.

\end{description}


(4) Incidental groundwater recharge data are stored in \code{inciden.rech},
an object of data.frame class with the following components:
\begin{description}

\item[EntityName] name of the irrigation entity.
\item[YearMonth] calendar year and month YYYYMM.
\item[SWDiv] surface-water diversions, in cubic meters per month.
\item[SeepFrac] canal seepage as a fraction of diversions, a dimensionless quantity.
\item[CanalSeep] canal seepage, in cubic meters per month.
\item[SWDel] surface-water delivered to field headgates, in cubic meters per month.
\item[area.sw] area irrigated by only surface water, in square meters.
\item[et.sw] evapotranspiration on lands irrigated by only surface water,
in cubic meters per month.
\item[precip.sw] precipitation on lands irrigated by only surface water,
in cubic meters per month.
\item[cir.sw] crop irrigation requirement on lands irrigated by only surface water,
in cubic meters per month.
\item[area.mix] area irrigated by both surface and groundwater, in square meters.
\item[et.mix] evapotranspiration on lands irrigated by both surface and groundwater,
in cubic meters per month.
\item[precip.mix] precipitation on lands irrigated by both surface and groundwater,
in cubic meters per month.
\item[cir.mix] crop irrigation requirement on lands irrigated by both surface and groundwater,
in cubic meters per month.
\item[area.gw] area irrigated by only groundwater, in square meters.
\item[et.gw] evapotranspiration on lands irrigated by only groundwater,
in cubic meters per month.
\item[precip.gw] precipitation on lands irrigated by only groundwater,
in cubic meters per month.
\item[cir.gw] crop irrigation requirement on lands irrigated by only groundwater,
in cubic meters per month.
\item[Eff] irrigation efficiency, a dimensionless quantity.
\item[GWDiv] recorded groundwater diversions, in cubic meters per month.
\item[WWDiv] inflow to municipal wastewater treatment plants, in cubic meters per month.
\item[hg.sw] surface-water delivered to field headgates on lands irrigated by only surface water,
in cubic meters per month.
\item[hg.mix] surface-water delivered to field headgates on lands irrigation by both surface and groundwater,
in cubic meters per month.
\item[rech.sw] incidental groundwater recharge beneath lands irrigated by only surface water,
in cubic meters per month.
\item[gw.dem.mix] groundwater demand on lands irrigated by both surface and groundwater,
in cubic meters per month.
\item[gw.div.est] calculated groundwater diversions, in cubic meters per month.
\item[rech.mix] incidental groundwater recharge beneath lands irrigated by both surface and groundwater,
in cubic meters per month.
\item[gw.only] groundwater demand on lands irrigated by only groundwater in entities with
lands also irrigated by both surface and groundwater, in cubic meters per month.
\item[rech.muni] incidental groundwater recharge beneath entities with
lands irrigated by only groundwater and lands irrigated by both surface and groundwater,
in cubic meters per month.
\item[gw.dem.gw] groundwater demand on lands irrigated by only groundwater in
entities without surface-water irrigation, in cubic meters per month.
\item[rech.gw] incidental groundwater recharge beneath lands irrigated by only groundwater,
in cubic meters per month.
\item[area.model] area of the irrigation entity that is located in the model domain,
in square meters.

\end{description}

Volumetric flow rates are calculated for their respective area in
the irrigation entity---not just that part overlying the model area.
Flow rate values are given this way in order to facilitate with quality assurance of
the water-budget calculation.
To calculate a simulated volumetric-flow rate: divide the flow rate by the affected area,
and then multiply this value by the area of the irrigation entity that is located in
the model domain (\code{area.model}).

(5) Well pumping data are also stored in \code{pumping.rech} (see \code{pod.rech} component),
an object of data.frame class with the following components:
\begin{description}

\item[WMISNumber] unique number assigned to a water right point of diversion.
\item[YearMonth] calendar year and month YYYYMM.
\item[Pumping] volumetric rate of pumping, in cubic meters per month.

\end{description}

\end{Value}
%
\begin{Author}\relax
J.C. Fisher, U.S. Geological Survey, Idaho Water Science Center

J. Sukow and M. McVay, Idaho Department of Water Resources
\end{Author}
%
\begin{SeeAlso}\relax
\code{\LinkA{UpdateWaterBudget}{UpdateWaterBudget}}
\end{SeeAlso}
%
\begin{Examples}
\begin{ExampleCode}
## Not run: # see Appendix A. Package Introduction

\end{ExampleCode}
\end{Examples}
\inputencoding{utf8}
\HeaderA{seepage.study}{Stream Seepage Study}{seepage.study}
\keyword{datasets}{seepage.study}
%
\begin{Description}\relax
A Wood River Valley stream seepage study with streamflow measurements made during
the months of August 2012, October 2012, and March 2013.
\end{Description}
%
\begin{Usage}
\begin{verbatim}
seepage.study
\end{verbatim}
\end{Usage}
%
\begin{Format}
An object of SpatialPointsDataFrame class containing 82 points
with the following variables:
\begin{description}

\item[Order] index used to preserve the downstream order of measurement sites.
\item[Name] local name for the measurement site.
\item[SiteNo] unique identifier for the measurement site within the
National Water Information System (NWIS).
\item[Type] the type of measurement site:
``Big Wood River'', ``Willow Creek'', ``Spring fed creeks'',
``Silver Creek'', ``Diversion'', ``Exchange well inflow'',
``Return'', and ``Tributary''.
\item[Comments] abbreviated site name
\item[Aug] volumetric flow rate measured during August 2012, in cubic meters per day.
\item[Oct] volumetric flow rate measured during October 2012, in cubic meters per day.
\item[Mar] volumetric flow rate measured during March 2013, in cubic meters per day.

\end{description}

Geographic coordinates are in units of meters, in conformance with the
North American Datum of 1983 (NAD 83), and placed in the
Idaho Transverse Mercator projection (\Rhref{https://www.idwr.idaho.gov/GIS/IDTM/}{IDTM}).
\end{Format}
%
\begin{Source}\relax
Derived from Bartolino (2014) seepage study,
Idaho Department of Water Resources, Water District 37 and 37M flow records.
\end{Source}
%
\begin{References}\relax
Bartolino, J.R., 2014, Stream seepage and groundwater levels, Wood River Valley,
south-central Idaho, 2012--13: U.S. Geological Survey Scientific Investigations Report 2014-5151,
34 p., \url{https://dx.doi.org/10.3133/sir20145151}.
\end{References}
%
\begin{Examples}
\begin{ExampleCode}
sp::plot(seepage.study)
str(seepage.study@data)

\end{ExampleCode}
\end{Examples}
\inputencoding{utf8}
\HeaderA{sensitivity}{PEST Sensitivity}{sensitivity}
\keyword{datasets}{sensitivity}
%
\begin{Description}\relax
Calibrated parameter values and composite sensitivities generated by PEST.
\end{Description}
%
\begin{Usage}
\begin{verbatim}
sensitivity
\end{verbatim}
\end{Usage}
%
\begin{Format}
An object of data.frame class with 336 records and the following variables:
\begin{description}

\item[parameter.desc] parameter description
\item[ID] unique identifier within the parameter type,
such as an identifier for a pilot point or irrigation entity.
\item[units] parameter units
\item[start.value] starting parameter value prior to model calibration.
\item[lower.bound] lower bound placed on the parameter value
during the model-calibration process.
\item[upper.bound] upper bound placed on the parameter value
during the model-calibration process.
\item[parameter.name] \Rhref{http://www.pesthomepage.org/}{PEST} parameter name
\item[group] PEST parameter group
\item[value] calibrated parameter value estimated by PEST.
\item[comp.sens] composite sensitivity generated during the final iteration of PEST.
\item[rel.comp.sens] relative composite sensitivity

\end{description}
\end{Format}
%
\begin{Source}\relax
Idaho Department of Water Resources, accessed on January 15, 2016
\end{Source}
%
\begin{SeeAlso}\relax
\code{\LinkA{pilot.points}{pilot.points}}, \code{\LinkA{irr.entities}{irr.entities}},
\code{\LinkA{river.reaches}{river.reaches}}, \code{\LinkA{drains}{drains}}, \code{\LinkA{tributaries}{tributaries}}
\end{SeeAlso}
%
\begin{Examples}
\begin{ExampleCode}
str(sensitivity)

\end{ExampleCode}
\end{Examples}
\inputencoding{utf8}
\HeaderA{soils}{Soil Units}{soils}
\keyword{datasets}{soils}
%
\begin{Description}\relax
Representation of mapped surficial soil units created by the Idaho Office of the
National Resource Conservation Service (NRCS).
Soils have been assigned a percolation rate based on the average,
saturated hydraulic conductivity of the soils as classified using the
Unified Soil Classification System (USCS).
\end{Description}
%
\begin{Usage}
\begin{verbatim}
soils
\end{verbatim}
\end{Usage}
%
\begin{Format}
An object of SpatialPolygonsDataFrame class containing 718 Polygons and a
data.frame with the following variables:
\begin{description}

\item[GroupSymbol] soil class identifier
\item[SoilLayer] identifier used to differentiate the soil data source
used to create the soils map.
Data sources are either `USCS' or `STATSGO',
the NRCS State Soil Geographic Data Base.
\item[SoilClass] description of the soil class.
\item[MinRate] lower percolation rate limit for the soil class, in meters per month.
\item[MaxRate] upper percolation rate limit for the soil class, in meters per month.
\item[PercolationRate] percolation rate in meters per month.

\end{description}

Geographic coordinates are in units of meters, in conformance with the
North American Datum of 1983 (NAD 83), and placed in the
Idaho Transverse Mercator projection (\Rhref{https://www.idwr.idaho.gov/GIS/IDTM/}{IDTM}).
\end{Format}
%
\begin{Source}\relax
Idaho Department of Water Resources, accessed on April 22, 2015
\end{Source}
%
\begin{Examples}
\begin{ExampleCode}
sp::spplot(soils, "PercolationRate")
str(soils@data)

\end{ExampleCode}
\end{Examples}
\inputencoding{utf8}
\HeaderA{streamgages}{Streamgages}{streamgages}
\keyword{datasets}{streamgages}
%
\begin{Description}\relax
Select streamgages in the Wood River Valley.
\end{Description}
%
\begin{Usage}
\begin{verbatim}
streamgages
\end{verbatim}
\end{Usage}
%
\begin{Format}
An object of SpatialPointsDataFrame class containing 9 points and a
data.frame with the following variable:
\begin{description}

\item[SiteNo] unique site number for the streamgage.
\item[SiteName] official name of the streamgage.

\end{description}

Geographic coordinates are in units of meters, in conformance with the
North American Datum of 1983 (NAD 83), and placed in the
Idaho Transverse Mercator projection (\Rhref{https://www.idwr.idaho.gov/GIS/IDTM/}{IDTM}).
\end{Format}
%
\begin{Source}\relax
National Water Information System (\Rhref{https://waterdata.usgs.gov/nwis}{NWIS}),
accessed on May 29, 2015.
\end{Source}
%
\begin{Examples}
\begin{ExampleCode}
str(streamgages)

\end{ExampleCode}
\end{Examples}
\inputencoding{utf8}
\HeaderA{streams.rivers}{Streams and Rivers}{streams.rivers}
\keyword{datasets}{streams.rivers}
%
\begin{Description}\relax
Streams and rivers of the Wood River Valley and surrounding areas.
\end{Description}
%
\begin{Usage}
\begin{verbatim}
streams.rivers
\end{verbatim}
\end{Usage}
%
\begin{Format}
An object of SpatialLinesDataFrame class containing 581 Lines.
Geographic coordinates are in units of meters, in conformance with the
North American Datum of 1983 (NAD 83), and placed in the
Idaho Transverse Mercator projection (\Rhref{https://www.idwr.idaho.gov/GIS/IDTM/}{IDTM}).
\end{Format}
%
\begin{Source}\relax
Idaho Department of Water Resources
(\Rhref{https://research.idwr.idaho.gov/index.html#GIS-Data}{IDWR}),
accessed on April 2, 2014
\end{Source}
%
\begin{Examples}
\begin{ExampleCode}
sp::plot(streams.rivers, col = "#3399CC")
str(streams.rivers@data)

\end{ExampleCode}
\end{Examples}
\inputencoding{utf8}
\HeaderA{subreach.recharge}{Stream-Aquifer Flow Exchange Along River Subreaches}{subreach.recharge}
\keyword{datasets}{subreach.recharge}
%
\begin{Description}\relax
Stream-aquifer flow exchange along river subreaches specified as aquifer recharge.
Values used as observations in parameter estimation.
\end{Description}
%
\begin{Usage}
\begin{verbatim}
subreach.recharge
\end{verbatim}
\end{Usage}
%
\begin{Format}
An object of data.frame class with 19 records and the following variables:
\begin{description}

\item[ReachNo] subreach number identifier
\item[Reach] name of the subreach.
\item[BigReachNo] reach number identifier
\item[BigReach] name of the reach.
\item[Aug] estimated volumetric flow rate measured during August 2012,
in cubic meters per day.
\item[Oct] estimated volumetric flow rate measured during October 2012,
in cubic meters per day.
\item[Mar] estimated volumetric flow rate measured during March 2013,
in cubic meters per day.

\end{description}
\end{Format}
%
\begin{Details}\relax
A positive stream-aquifer flow exchange indicates aquifer recharge
(also know as a losing river subreach).
\end{Details}
%
\begin{Source}\relax
Flow values calculated from \code{\LinkA{seepage.study}{seepage.study}} data.
\end{Source}
%
\begin{Examples}
\begin{ExampleCode}
str(subreach.recharge)

\end{ExampleCode}
\end{Examples}
\inputencoding{utf8}
\HeaderA{swe}{Snow Water Equivalent}{swe}
\keyword{datasets}{swe}
%
\begin{Description}\relax
Average daily snow water equivalent (SWE) at weather stations in the
Wood River Valley and surrounding areas.
\end{Description}
%
\begin{Usage}
\begin{verbatim}
swe
\end{verbatim}
\end{Usage}
%
\begin{Format}
An object of data.frame class with 366 records and the following variables:
\begin{description}

\item[MonthDay] month and day, with a required date format of \code{MMDD}.
\item[Choco] daily SWE recorded at the Chocolate Gulch snow telemetry (SNOTEL)
weather station.
\item[Hailey] daily SWE recorded at the Hailey Ranger Station at Hailey
hydrometeorological automated data system (HADS) weather station.
\item[Picabo] daily SWE recorded at the Picabo PICI HADS weather station.

\end{description}
\end{Format}
%
\begin{Source}\relax
Idaho Department of Water Resources, accessed on April 24, 2015
\end{Source}
%
\begin{SeeAlso}\relax
\code{\LinkA{weather.stations}{weather.stations}}, \code{\LinkA{precip.zones}{precip.zones}},
\code{\LinkA{precipitation}{precipitation}}
\end{SeeAlso}
%
\begin{Examples}
\begin{ExampleCode}
str(swe)

\end{ExampleCode}
\end{Examples}
\inputencoding{utf8}
\HeaderA{tributaries}{Tributary Basin Underflow}{tributaries}
\keyword{datasets}{tributaries}
%
\begin{Description}\relax
Location and average flow conditions for model boundaries in the
major tributary canyons and upper part of the Wood River Valley, south-central Idaho.
\end{Description}
%
\begin{Usage}
\begin{verbatim}
tributaries
\end{verbatim}
\end{Usage}
%
\begin{Format}
An object of SpatialPolygonsDataFrame class containing a set of 22 Polygons and a
data.frame with the following variable:
\begin{description}

\item[Name] tributary name
\item[MinLSD] minimum land-surface datum (elevation) along the transect,
in meters above the North American Vertical Datum of 1988 (NAVD 88).
\item[BdrkDepth] mean saturated thickness along the transect line, in meters;
estimated as the distance between the estimated water table and bedrock elevations.
\item[TribWidth] width of the tributary canyon, or length of the transect line, in meters.
\item[LandGrad] land surface elevation gradient perpendicular to the
cross-sectional transect line, a dimensionless quantity.
\item[K] hydraulic conductivity in meters per day.
\item[SatArea] estimated saturated cross-sectional area, in square meters;
its geometry is represented as the lower-half of an ellipse with
width and height equal to \code{TribWidth} and \code{BdrkDepth}, respectively.
\item[DarcyFlow] groundwater flow rate, in cubic meters per day, calculated using a
\Rhref{https://en.wikipedia.org/wiki/Darcy_law}{Darcian} analysis.
\item[BasinArea] land-surface area defined by the basin above the cross-sectional transect line.
\item[BasinAreaType] label that describes the relative basin size; where
\code{"big"} indicates a basin area greater than 10 square miles (25.9 square kilometers), and
\code{"small"} indicates a basin area that is less than this breakpoint value.
\item[PrecipRate] mean precipitation rate within the basin area, in meters per day.
\item[PrecipFlow] mean precipitation flow rate, in cubic meters per day,
calculated by multiplying \code{PrecipRate} by \code{BasinArea}.
\item[FlowRatio] ratio of darcy flow rate to precipitation flow rate,
or \code{DarcyFlow} divided by \code{PrecipFlow}, a dimensionless quantity.
\item[Flow] estimated average volumetric flow rate, in cubic meters per day.

\end{description}

Geographic coordinates are in units of meters, in conformance with the
North American Datum of 1983 (NAD 83), and placed in the
Idaho Transverse Mercator projection (\Rhref{https://www.idwr.idaho.gov/GIS/IDTM/}{IDTM}).
\end{Format}
%
\begin{Source}\relax
U.S. Geological Survey, accessed on July 2, 2015;
a Keyhole Markup Language (\Rhref{https://en.wikipedia.org/wiki/Kml}{KML}) file created in
\Rhref{https://www.google.com/earth/}{Google Earth} with polygons drawn by hand in
areas of known specified flow boundaries.
Transect lines are defined by the polygon boundaries within the extent of
alluvium aquifer (see \code{\LinkA{alluvium.extent}{alluvium.extent}} dataset).
The land surface gradient (\code{LandGrad}) was estimated from the
\code{\LinkA{land.surface}{land.surface}} dataset.
Hydraulic conductivity (\code{K}) is the average of two geometric means of hydraulic conductivity
in the unconfined aquifer taken from table 2 in Bartolino and Adkins (2012).
The U.S. Geologic Survey \Rhref{https://water.usgs.gov/osw/streamstats/}{StreamStats} tool
(Ries and others, 2004) was used to delineate the basin area (\code{BasinArea}) and
estimate the precipitation rate (\code{PrecipRate}).
See the appendix C. Package Dataset Creation for the \R{} code used to calculate the
flow estimates (\code{Flow}).
\end{Source}
%
\begin{References}\relax
Bartolino, J.R., and Adkins, C.B., 2012, Hydrogeologic framework of the
Wood River Valley aquifer system, south-central Idaho: U.S. Geological Survey
Scientific Investigations Report 2012-5053, 46 p.,
available at \url{https://pubs.usgs.gov/sir/2012/5053/}.

Ries, K.G., Steeves, P.A., Coles, J.D., Rea, A.H., and Stewart, D.W., 2004,
StreamStats--A U.S. Geological Survey web application for stream information:
U.S. Geological Survey Fact Sheet FS-2004-3115, 4 p., available at
\url{https://pubs.er.usgs.gov/publication/fs20043115}.
\end{References}
%
\begin{Examples}
\begin{ExampleCode}
sp::plot(tributaries, border = "red")
sp::plot(alluvium.extent, add = TRUE)
str(tributaries@data)

\end{ExampleCode}
\end{Examples}
\inputencoding{utf8}
\HeaderA{tributary.streams}{Tributary Streams}{tributary.streams}
\keyword{datasets}{tributary.streams}
%
\begin{Description}\relax
Tributary streams of the upper Wood River Valley and surrounding areas.
\end{Description}
%
\begin{Usage}
\begin{verbatim}
tributary.streams
\end{verbatim}
\end{Usage}
%
\begin{Format}
An object of SpatialLinesDataFrame class containing 88 Lines.
Geographic coordinates are in units of meters, in conformance with the
North American Datum of 1983 (NAD 83), and placed in the
Idaho Transverse Mercator projection (\Rhref{https://www.idwr.idaho.gov/GIS/IDTM/}{IDTM}).
\end{Format}
%
\begin{Source}\relax
Idaho Department of Water Resources, accessed on June 1, 2015
\end{Source}
%
\begin{Examples}
\begin{ExampleCode}
sp::plot(tributary.streams, col = "#3399CC")
str(tributary.streams@data)

\end{ExampleCode}
\end{Examples}
\inputencoding{utf8}
\HeaderA{UpdateWaterBudget}{Update Water Budget}{UpdateWaterBudget}
\keyword{utilities}{UpdateWaterBudget}
%
\begin{Description}\relax
This function determines the specified-flow boundary conditions for the groundwater-flow model.
These boundary conditions include:
(1) natural and incidental groundwater recharge at the water table,
(2) groundwater pumping at production wells, and
(3) groundwater underflow in the major tributary valleys.
Specified-flow values are saved to disk by rewriting the
\Rhref{https://water.usgs.gov/ogw/modflow/}{MODFLOW} Well Package file (\file{.wel}).
Note that this function is executed after each iteration of \Rhref{http://www.pesthomepage.org/}{PEST}.
\end{Description}
%
\begin{Usage}
\begin{verbatim}
UpdateWaterBudget(dir.run, id, qa.tables = c("none", "si", "english"),
  ss.interval = NULL, iwelcb = ifelse(interactive(), 50L, 0L),
  canal.seep = wrv::canal.seep, efficiency = wrv::efficiency,
  gage.disch = wrv::gage.disch, pod.wells = wrv::pod.wells,
  tributaries = wrv::tributaries, ...)
\end{verbatim}
\end{Usage}
%
\begin{Arguments}
\begin{ldescription}
\item[\code{dir.run}] character.
Path name of the directory to read/write model files.

\item[\code{id}] character.
Short identifier (file name) for model files.

\item[\code{qa.tables}] character.
Indicates if quality assurance tables are written to disk;
by default "none" of these tables are written.
Values of "si" and "english" indicate that table values are written in
metric and English units, respectively.

\item[\code{ss.interval}] Date or character.
Vector of length 2 specifying the start and end dates for the period used to
represent steady-state boundary conditions.
That is, recharge values for stress periods coinciding with this time period are
averaged and used as steady-state boundary conditions.
The required date format is YYYY-MM-DD.
This argument overrides the \code{ss.stress.periods} object in the \file{model.rda} file,
see `Details' section for additional information.

\item[\code{iwelcb}] integer.
A flag and unit number.
If equal to zero, the default, cell-by-cell flow terms resulting from
conditions in the MODFLOW Well Package will not be written to disk.
A value of 0 is appropriate for model calibration,
where MODFLOW run times are kept as short as possible.
If greater than zero, the cell-by-cell flow terms are written to disk.
See the MODFLOW Name File (\file{*.nam}) for the unit number associated with
the budget file (\file{*.bud}).
The default value is 50 (value specified in the \code{\LinkA{WriteModflowInput}{WriteModflowInput}} function)
if \R{} is being used interactively and 0 otherwise.

\item[\code{canal.seep}] data.frame.
See \code{\LinkA{canal.seep}{canal.seep}} dataset for details.

\item[\code{efficiency}] data.frame.
See \code{\LinkA{efficiency}{efficiency}} dataset for details.

\item[\code{gage.disch}] data.frame.
See \code{\LinkA{gage.disch}{gage.disch}} dataset for details.

\item[\code{pod.wells}] SpatialPointsDataFrame.
See \code{\LinkA{pod.wells}{pod.wells}} dataset for details.

\item[\code{tributaries}] SpatialPolygonsDataFrame.
See \code{\LinkA{tributaries}{tributaries}} dataset for details.

\item[\code{...}] Arguments to be passed to \code{\LinkA{RunWaterBalance}{RunWaterBalance}}, such as evapotranspiration \code{et}.
\end{ldescription}
\end{Arguments}
%
\begin{Details}\relax
Files read during execution, and located within the \code{dir.run} directory,
inlcude the MODFLOW hydraulic conductivity reference files \file{hk1.ref},
\file{hk2.ref}, and \file{hk3.ref} corresponding to model layers 1, 2, and 3, respectively.
Hydraulic conductivity values are read from a two-dimensional array in
matrix format with `white-space' delimited fields.
And a binary data file \file{model.rda} containing the following serialized \R{} objects:
\code{rs}, \code{misc}, \code{trib}, \code{tr.stress.periods}, and \code{ss.stress.periods}.

\code{rs} is an object of RasterStack class with raster layers ``lay1.top'',
``lay1.bot'', ``lay2.bot'', and ``lay3.bot''.
These raster layers describe the geometry of the model grid; that is,
the upper and lower elevation of model layer 1, and the bottom elevations of model layers 2 and 3.
Missing cell values (equal to NA) indicate inactive model cells lying outside of the model domain.

\code{misc} is a data.frame object with miscellaneous seepage,
such as from the `Bellevue Waste Water Treatment Plant ponds' and the `Bypass Canal'.
This object is comprised of the following components:
\bold{lay}, \bold{row}, \bold{col} are integer values specifying a
model cell's layer, row, and column index, respectively; and
\bold{ss}, \bold{199501}, \bold{199502}, \dots, \bold{201012} are numeric values of
elevation during each stress period, respectively,
in meters above the North American Vertical Datum of 1988.

\code{trib} is a data.frame object with default values for the
long-term mean underflows in each of the tributary basins.
The object is comprised of the following components:
\bold{Name} is a unique identifier for the tributary basin;
\bold{lay}, \bold{row}, \bold{col} are \code{integer} values of a
model cell's layer, row, and column index, respectively; and
\bold{ss}, \bold{199501}, \bold{199502}, \dots, \bold{201012} are numeric values of
underflow during each stress period, respectively, in cubic meters per day.

\code{tr.stress.periods} is a vector of Date values giving the start and end dates for
stress periods in the model simulation period (1995--2010).

\code{ss.stress.periods} is a vector of Date values giving the start and end dates for
stress periods used to define steady-state conditions.

\code{reduction} is a numeric default value for the signal amplitude reduction algorithm,
a dimensionless quantity.

\code{d.in.mv.ave} is a numeric default value for the number of days in the
moving average subset.
\end{Details}
%
\begin{Value}
Returns an object of difftime class, the runtime for this function.
Used for the side-effect of files written to disk.

A MODFLOW Well Package file \file{<id>.wel} is always written to disk; whereas,
parameter estimation files \file{seep.csv}, \file{eff.csv}, and \file{trib.csv}, and
a script file \file{UpdateBudget.bat}, are only written if they do not already exist.
The script file may be used to automate the execution of this function from a
file manager (such as, Windows Explorer).

The \file{seep.csv} file stores as tabular data the canal seepage fraction for
each of the irrigation entities.
Its character and numeric data fields are delimited by commas (a comma-separated-value [CSV] file).
The first line is reserved for field names ``EntityName'' and ``SeepFrac''.

The \file{eff.csv} file stores as tabular data the irrigation efficiency for
each of the irrigation entities.
Its character and numeric data fields are delimited by commas.
The first line is reserved for field names ``EntityName'' and ``Eff''.

The \file{trib.csv} file stores as tabular data the underflow boundary conditions for
each tributary basin.
Its character and numeric data fields are delimited by commas.
The first line is reserved for field names ``Name'' and ``Value''.
Data records include a long-term mean flow multiplier for
each of the tributary basins (name is the unique identifier for the tributary),
a record for the amplitude reduction (\code{reduction}), and
a record for the number of days in the moving average (\code{d.in.mv.ave}).

If the \code{qa.tables} argument is specified as either ``si'' or ``english'',
quality assurance tables are written to disk as CSV files (\file{qa-*.csv}).
Volumetric flow rate data within these tables is described in the
`Value' section of the \code{\LinkA{RunWaterBalance}{RunWaterBalance}} function;
see returned list components \code{natural.rech}, \code{inciden.rech}, and \code{pumping.rech}.
The well configuration data are described in the `Value' section of the
\code{\LinkA{GetWellConfig}{GetWellConfig}} function.
\end{Value}
%
\begin{Author}\relax
J.C. Fisher, U.S. Geological Survey, Idaho Water Science Center
\end{Author}
%
\begin{SeeAlso}\relax
\code{\LinkA{RunWaterBalance}{RunWaterBalance}}, \code{\LinkA{GetSeasonalMult}{GetSeasonalMult}}
\end{SeeAlso}
%
\begin{Examples}
\begin{ExampleCode}
## Not run: 
  dir.run <- file.path(getwd(), "model/model1")
  UpdateWaterBudget(dir.run, "wrv_mfusg", qa.tables = "si",
                    ss.interval = c("1998-01-01", "2011-01-01"))

## End(Not run)

\end{ExampleCode}
\end{Examples}
\inputencoding{utf8}
\HeaderA{weather.stations}{Weather Stations}{weather.stations}
\keyword{datasets}{weather.stations}
%
\begin{Description}\relax
Weather stations in the Wood River Valley and surrounding areas.
\end{Description}
%
\begin{Usage}
\begin{verbatim}
weather.stations
\end{verbatim}
\end{Usage}
%
\begin{Format}
An object of SpatialPointsDataFrame class containing 5 points
and the following variables:
\begin{description}

\item[name] name of the weather station.
\item[id] unique identifier for the weather station.
\item[type] type of weather stations:
\code{"HADS"}, a Hydrometeorological Automated Data System operated by the
National Weather Service Office of Dissemination;
\code{"AgriMet"}, a satellite-telemetry network of
automated agricultural weather stations operated and
maintained by the Bureau of Reclamation; and
\code{"SNOTEL"}, an automated system of snowpack and
related climate sensors operated by the
Natural Resources Conservation Service.
\item[organization] is the managing organization.
\item[elevation] is the elevation of the weather station in
meters above the North American Vertical Datum of 1988 (NAVD 88).

\end{description}

Geographic coordinates are in units of meters, in conformance with the
North American Datum of 1983 (NAD 83), and placed in the
Idaho Transverse Mercator projection (\Rhref{https://www.idwr.idaho.gov/GIS/IDTM/}{IDTM}).
\end{Format}
%
\begin{Source}\relax
National Oceanic and Atmospheric Administration (NOAA),
Bureau of Reclamation, Natural Resources Conservation Service (NRCS),
accessed on May 1, 2015
\end{Source}
%
\begin{Examples}
\begin{ExampleCode}
sp::plot(alluvium.extent)
sp::plot(weather.stations, col = "red", add = TRUE)
str(weather.stations@data)

\end{ExampleCode}
\end{Examples}
\inputencoding{utf8}
\HeaderA{wetlands}{Wetlands}{wetlands}
\keyword{datasets}{wetlands}
%
\begin{Description}\relax
Wetlands in the Wood River Valley and surrounding areas.
\end{Description}
%
\begin{Usage}
\begin{verbatim}
wetlands
\end{verbatim}
\end{Usage}
%
\begin{Format}
An object of SpatialPolygons class containing 3,024 Polygons.
Geographic coordinates are in units of meters, in conformance with the
North American Datum of 1983 (NAD 83), and placed in the
Idaho Transverse Mercator projection (\Rhref{https://www.idwr.idaho.gov/GIS/IDTM/}{IDTM}).
\end{Format}
%
\begin{Source}\relax
U.S. Fish and Wildlife Service National Wetlands Inventory,
accessed on April 2, 2014
\end{Source}
%
\begin{Examples}
\begin{ExampleCode}
sp::plot(wetlands, col = "#CCFFFF", border = "#3399CC", lwd = 0.5)
print(wetlands)

\end{ExampleCode}
\end{Examples}
\inputencoding{utf8}
\HeaderA{wl.200610}{Groundwater-Level Contours for October 2006}{wl.200610}
\keyword{datasets}{wl.200610}
%
\begin{Description}\relax
Groundwater-level contours with a 20 foot (6.096 meter) contour interval for the
unconfined aquifer in the Wood River Valley, south-central Idaho,
representing conditions during October 2006.
\end{Description}
%
\begin{Usage}
\begin{verbatim}
wl.200610
\end{verbatim}
\end{Usage}
%
\begin{Format}
An object of SpatialLinesDataFrame class containing 265 Lines and a
data.frame with the following variables:
\begin{description}

\item[CONTOUR] groundwater elevation contour value in meters above the
North American Vertical Datum of 1988 (NAVD 88).
\item[certainty] certainty of the groundwater-level contour based on
data position and density, specified as \code{"good"} or \code{"poor"}.

\end{description}

Geographic coordinates are in units of meters, in conformance with the
North American Datum of 1983 (NAD 83), and placed in the
Idaho Transverse Mercator projection (\Rhref{https://www.idwr.idaho.gov/GIS/IDTM/}{IDTM}).
\end{Format}
%
\begin{Source}\relax
This dataset is from Plate 1 in Skinner and others (2007), and available on the
\Rhref{https://water.usgs.gov/GIS/metadata/usgswrd/XML/sir2007-5258_oct2006wl.xml}{WRD NSDI Node}.
\end{Source}
%
\begin{References}\relax
Skinner, K.D., Bartolino, J.R., and Tranmer, A.W., 2007,
Water-resource trends and comparisons between partial development and
October 2006 hydrologic conditions, Wood River Valley, south-central, Idaho:
U.S. Geological Survey Scientific Investigations Report 2007-5258, 30 p.,
available at \url{https://pubs.usgs.gov/sir/2007/5258/}
\end{References}
%
\begin{Examples}
\begin{ExampleCode}
is.good <- wl.200610@data$certainty == "good"
sp::plot(wl.200610[is.good, ], col = "blue")
sp::plot(wl.200610[!is.good, ], col = "red", lty = 2, add = TRUE)
str(wl.200610@data)

\end{ExampleCode}
\end{Examples}
\inputencoding{utf8}
\HeaderA{WriteModflowInput}{Write MODFLOW Input Files}{WriteModflowInput}
\keyword{IO}{WriteModflowInput}
%
\begin{Description}\relax
This function generates and writes input files for a MODFLOW simulation of
groundwater flow in the Wood River Valley (WRV) aquifer system.
\end{Description}
%
\begin{Usage}
\begin{verbatim}
WriteModflowInput(rs.model, rech, well, trib, misc, river, drain, id, dir.run,
  is.convertible = FALSE, ss.perlen = 0L, tr.stress.periods = NULL,
  ntime.steps = 4L, mv.flag = 1e+09, auto.flow.reduce = FALSE,
  verbose = TRUE)
\end{verbatim}
\end{Usage}
%
\begin{Arguments}
\begin{ldescription}
\item[\code{rs.model}] RasterStack.
Collection of RasterLayer objects with the same extent and resolution,
see `Details' for required raster layers.

\item[\code{rech}] data.frame.
Areal recharge rate, in cubic meters per day.
Variables describe the model cell location (\code{lay}, \code{row}, \code{col}) and
volumetric rate during each stress period
(\code{ss}, \code{199501}, \code{199502}, \dots, \code{201012}).

\item[\code{well}] data.frame.
Well pumping at point locations in cubic meters per day.
Variables describe the model cell location and volumetric rate during each stress period.

\item[\code{trib}] data.frame.
Incoming flows from the major tributary canyons.
Variables describe the model cell location and volumetric rate during each stress period.

\item[\code{misc}] data.frame.
Direct recharge from miscellaneous seepage sites in cubic meters per day.
Variables describe the model cell location and volumetric rate during each stress period.

\item[\code{river}] data.frame.
River conditions.
Variables describe the model cell location, river conductance
(\code{cond}) in square meters per day, river bottom elevation (\code{bottom}) in
meters above the North American Vertical Datum of 1988 (NAVD 88), and
a numeric river reach identifier (\code{id}).

\item[\code{drain}] data.frame.
Drain conditions for groundwater outlet boundaries.
Variables describe the model cell location, drain threshold elevation
(\code{elev}) in meters above the NAVD 88, drain conductance (\code{cond}) in
square meters per day, and a numeric identifier (\code{id}) indicating the
drains general location.

\item[\code{id}] character.
Short identifier for the model run.

\item[\code{dir.run}] character.
Path name of the directory to write model input files.

\item[\code{is.convertible}] logical.
If true, indicates model layers are `convertible', with
transmissivity computed using upstream water-table depth.
Otherwise, model layers are `confined' and transmissivity is constant over time.

\item[\code{ss.perlen}] integer or difftime.
Length of the steady-state stress period in days.

\item[\code{tr.stress.periods}] Date.
Vector of start times for each stress period in the transient simulation.
If missing, only steady-state conditions are simulated.

\item[\code{ntime.steps}] integer.
Number of uniform time steps in a stress period.

\item[\code{mv.flag}] numeric.
Missing value flag for output reference data files.

\item[\code{auto.flow.reduce}] logical.
If true, a simulated well will adjust pumping according to
supply under bottom-hole conditions.
Pumping rates that have been automatically reduced will be written to a
model output file (\file{.afr}).

\item[\code{verbose}] logical.
If true, additional information is written to the
listing file (\file{.lst}) and budget file (\file{.bud})
\end{ldescription}
\end{Arguments}
%
\begin{Details}\relax
Groundwater flow in the WRV aquifer system is simulated using the
\Rhref{https://water.usgs.gov/ogw/mfusg/}{MODFLOW-USG} groundwater-flow model.
This numerical model was chosen for its ability to solve
complex unconfined groundwater flow simulations.
The solver implemented in MODFLOW-USG incorporates the Newton-Raphson formulation for
improving solution convergence and avoiding problems with the drying and
rewetting of cells (Niswonger and others, 2011).
A structured finite-difference grid is implemented in the model to
(1) simplify discretization,
(2) keep formats and structures for the MODFLOW-USG packages identical to those of
\Rhref{https://water.usgs.gov/ogw/modflow/MODFLOW-2005-Guide/index.html}{MODFLOW-2005}, and
(3) allow any MODFLOW post-processor to be used to analyze the results of the MODFLOW-USG simulation
(such as \Rhref{https://water.usgs.gov/nrp/gwsoftware/modelviewer/ModelViewer.html}{Model Viewer}).

Model input files are written to \code{dir.run} and include the following MODFLOW Package files:
Name (\file{.nam}), Basic (\file{.ba6}), Discretization (\file{.dis}),
Layer-Property Flow (\file{.lpf}), Drain (\file{.drn}), River (\file{.riv}),
Well (\file{.wel}), Sparse Matrix Solver (\file{.sms}), and Output Control (\file{.oc}).
See the users guide (\Cite{Description of Model Input and Output}) included with the MODFLOW-USG
software for details on input file formats and structures.

Data within the \code{rech}, \code{well}, \code{trib}, and \code{misc} arguments are
combined in the MODFLOW Well Package and identifiable with added \code{id} values of
1, 2, 3, and 4, respectively.

The Layer-Property Flow file includes options for the calculation of vertical flow in
partially dewatered cells.
For the WRV model, where there is no indication that perched conditions exist,
CONSTANTCV and NOVFC options are used to create the most stable solution
(Panday and others, 2013, p. 15-16).
Options for the Sparse Matrix Solver were set for unconfined simulations by
implementing an upstream-weighting scheme with Newton-Raphson linearization,
Delta-Bar-Delta under-relaxation, and the \eqn{\chi}{}MD solver of Ibaraki (2005).

The raster stack \code{rs.model} includes the following layers:
\begin{description}

\item[lay1.top] elevation at the top of model layer 1 (land surface),
in meters above the NAVD 88.
\item[lay1.bot] elevation at the bottom of model layer 1, in meters above the NAVD 88.
\item[lay2.bot] elevation at the bottom of model layer 2.
\item[lay3.bot] elevation at the bottom of model layer 3.
\item[lay1.strt] initial (starting) hydraulic head in model layer 1,
in meters above the NAVD 88.
\item[lay2.strt] initial hydraulic head in model layer 2.
\item[lay3.strt] initial hydraulic head in model layer 3.
\item[lay1.zones] hydrogeologic zones in model layer 1 where values
equal to 1 is unconfined alluvium, equal to 2 is basalt,
equal to 3 is clay, and equal to 4 is confined alluvium.
\item[lay2.zones] hydrogeologic zones in model layer 2.
\item[lay3.zones] hydrogeologic zones in model layer 3.
\item[lay1.hk] horizontal hydraulic conductivity in model layer 1, in meters per day.
\item[lay2.hk] horizontal hydraulic conductivity in model layer 2.
\item[lay3.hk] horizontal hydraulic conductivity in model layer 3.

\end{description}

\end{Details}
%
\begin{Value}
Used for the side-effect of files written to disk.
\end{Value}
%
\begin{Author}\relax
J.C. Fisher, U.S. Geological Survey, Idaho Water Science Center
\end{Author}
%
\begin{References}\relax
Ibaraki, M., 2005, \eqn{\chi}{}MD User's guide-An efficient sparse matrix solver library, version 1.30:
Columbus, Ohio State University School of Earth Sciences.

Niswonger, R.G., Panday, Sorab, and Ibaraki, Motomu, 2011, MODFLOW-NWT, A Newton formulation for MODFLOW-2005:
U.S. Geological Survey Techniques and Methods 6-A37, 44 p., available at \url{https://pubs.usgs.gov/tm/tm6a37/}.

Panday, Sorab, Langevin, C.D., Niswonger, R.G., Ibaraki, Motomu, and Hughes, J.D., 2013, MODFLOW-USG version 1:
An unstructured grid version of MODFLOW for simulating groundwater flow and tightly coupled processes using a
control volume finite-difference formulation: U.S. Geological Survey Techniques and Methods, book 6, chap. A45,
66 p., available at \url{https://pubs.usgs.gov/tm/06/a45/}.
\end{References}
%
\begin{Examples}
\begin{ExampleCode}
## Not run: # see Appendix D. Uncalibrated Groundwater-Flow Model

\end{ExampleCode}
\end{Examples}
\inputencoding{utf8}
\HeaderA{zone.properties}{Hydraulic Properties of Hydrogeologic Zones}{zone.properties}
\keyword{datasets}{zone.properties}
%
\begin{Description}\relax
Estimates of the hydraulic properties for each hydrogeologic zone.
\end{Description}
%
\begin{Usage}
\begin{verbatim}
zone.properties
\end{verbatim}
\end{Usage}
%
\begin{Format}
An object of \code{data.frame} class with the following variables:
\begin{description}

\item[ID] numeric identifier for the hydrogeologic zone.
\item[name] name of the hydrogeologic zone.
\item[vani] vertical anisotropy, a dimensionless quantity.
\item[sc] storage coefficient, a dimensionless quantity.
\item[sy] specific yield, a dimensionless quantity.
\item[hk] horizontal hydraulic conductivity in meters per day.
\item[ss] specific storage in inverse meter.

\end{description}
\end{Format}
%
\begin{References}\relax
Bartolino, J.R., and Adkins, C.B., 2012, Hydrogeologic framework of the
Wood River Valley aquifer system, south-central Idaho:
U.S. Geological Survey Scientific Investigations Report 2012-5053, 46 p.,
available at \url{https://pubs.usgs.gov/sir/2012/5053/}.
\end{References}
%
\begin{Examples}
\begin{ExampleCode}
str(zone.properties)

\end{ExampleCode}
\end{Examples}
\printindex{}
\end{document}
